\section{Busca de dados disponíveis para o Santa Luzia}

Em um levantamento realizado em 2018 pela Companhia de Planejamento do Distrito Federal(Codeplan) foi feito o estudo da Pesquisa Distrital por Amostra de Domicílio (PDAD), esse estudo tem como principal função atualizar o perfil socioeconômico dos moradores do Distrito Federal(DF), analisar possíveis carências na região urbana do DF e avaliar mudanças estudos anteriores. Nessa pesquisa foram visitados 21908 domicílios ao longo de 7 meses e observou-se que dentre as áreas abrangidas pelo estudo na região da Santa Luzia a quantidade de domicílios era 3.793 e que população de Santa Luzia contava com aproximadamente 16 mil moradores.\cite{CODEPLAN:online}

O programa de Pesquisa Distrital por Amostra de Domicilio(PDAD) foi elaborado e dividido em 4 fases. A primeira fase com foco em planejamento como elaboração de manuais, questionários. Logo após isso a segunda fase com um conceito pre-pesquisa com pré-testes e o treinamento dos pesquisadores. Na terceira fase, iniciou-se de fato a pesquisa em campo com os questionários nas residencias dos entrevistados. Por fim na quarta fase foi gerado o relatório e armazenamento no banco de dados assim como analises técnicas dos resultados e o relatório de resultados da PDAD\cite{CODEPLAN:online}

quais os dados? onde estão disponíveis? como estão organizados?
como foram produzidos?

\section{Busca de dados disponíveis para o RK}


quais os dados? como estão organizados? como foram produzidos?

\section{Simulação versos realidade}

\subsection{Simulação do santa luzia versus realidade dos dados}

\subsection{Simulação do RK versus realidade dos dados}


