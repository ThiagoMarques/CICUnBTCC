\section{Busca de dados disponíveis para o Santa Luzia}

Em um levantamento realizado em 2018 pela Companhia de Planejamento do Distrito Federal(Codeplan) foi feito o estudo da Pesquisa Distrital por Amostra de Domicílio (PDAD), esse estudo tem como principal função atualizar o perfil socioeconômico dos moradores do Distrito Federal(DF), analisar possíveis carências na região urbana do DF e avaliar mudanças estudos anteriores. Nessa pesquisa foram visitados 21908 domicílios ao longo de 7 meses e observou-se que dentre as áreas abrangidas pelo estudo na região da Santa Luzia a quantidade de domicílios era 3.793 e que população de Santa Luzia contava com aproximadamente 16 mil moradores.\cite{CODEPLAN:online}

A PDAD 2018 foi desenvolvida e realizada em quatro etapas. A primeira etapa foi dedicada ao
planejamento da pesquisa, quando foram definidas as áreas de abrangência, o cronograma de atividades,
a elaboração de manuais e do questionário a ser aplicado. Na segunda etapa, foram realizados a pesquisa
de pré-testes e o treinamento dos pesquisadores. Na terceira etapa, foram realizados a aplicação dos
questionários nos domicílios amostrados pela Codeplan e os serviços de checagem das informações
coletadas pelo Instituto Euvaldo Lodi — IEL, empresa contratada para a coleta dos dados. A quarta e
última etapa foi dedicada à formatação do banco de dados; realização das análises de consistência;
produção de tabulações e de análises técnicas de resultados; e, finalmente, à confecção do Relatório de
Resultados da PDAD 2018.\cite{CODEPLAN:online}





quais os dados? onde estão disponíveis? como estão organizados?
como foram produzidos?

\section{Busca de dados disponíveis para o RK}

quais os dados? como estão organizados? como foram produzidos?

\section{Simulação versos realidade}

\subsection{Simulação do santa luzia versus realidade dos dados}

\subsection{Simulação do RK versus realidade dos dados}
