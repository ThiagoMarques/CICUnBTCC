%\section{Busca de dados disponíveis para o Santa Luzia}

%Em um levantamento realizado em 2018 pela Companhia de Planejamento do Distrito Federal(Codeplan) foi feito o estudo da Pesquisa Distrital por Amostra de Domicílio (PDAD), esse estudo tem como principal função atualizar o perfil socioeconômico dos moradores do Distrito Federal(DF), analisar possíveis carências na região urbana do DF e avaliar mudanças estudos anteriores. Nessa pesquisa foram visitados 21908 domicílios ao longo de 7 meses e observou-se que dentre as áreas abrangidas pelo estudo na região da Santa Luzia a quantidade de domicílios era 3.793 e que população de Santa Luzia contava com aproximadamente 16 mil moradores.\cite{CODEPLAN:online}

%O programa de Pesquisa Distrital por Amostra de Domicilio (PDAD) foi elaborado e dividido em 4 fases. A primeira fase com foco em planejamento como elaboração de manuais, questionários. A segunda fase com um conceito pré-pesquisa com pré-testes e o treinamento dos pesquisadores. Na terceira fase, iniciou-se de fato a pesquisa em campo com os questionários nas residencias dos entrevistados. Por fim na quarta fase foi gerado o relatório e armazenamento no banco de dados assim como analises técnicas dos resultados e o relatório de resultados da PDAD\cite{CODEPLAN:online}

%Os dados da Coraci é o representante da parte social/civil e está em contato direto com a Fiocruz.

\section{Busca de dados disponíveis para o RK}

Os dados do condomínio RK foram cedidas pela equipe do Centro de Estudos Ambientais do Condomínio Rural Residencial RK (CEA/RK), ligado à administração do condomínio, com o objetivo de controlar e reduzir riscos à saúde da comunidade \cite{CondominioRK:online}. 

\subsubsection{Dados Compilados}


Foram disponibilizados os seguintes dados compilados do condomínio:


\begin{itemize}
\item População absoluta: aproximadamente 8000 moradores;
\item Idade média: 35 anos;
\item Idade máxima: 93 anos;
\item Média de pessoas por casa: 4 habitantes;
\item Tamanho da área: 1.481km²
\item Quantidade de casas: 1900 casas;
\item Quantidade de comércios: 41 imóveis comerciais;
\item Densidade demográfica preliminar (habitantes/km²): aproximadamente 5,4 hab/km²
\end{itemize}

Abaixo, os dados de perfil etário dos moradores em um gráfico:

\figura{4_1_perfil_etário_RK.PNG}{Perfil etário dos moradores do Condomínio RK}{basicmodel}{width=0.75\textwidth}

\subsubsection{Dados geográficos}

Os dados geográficos cedidos pelo Centro de Estudos Ambientais do Condomínio Rural Residencial RK (CEA/RK) foram enviados no formato shapefile, conforme imagem abaixo:

\figura{5_1_shapefile_RK.PNG}{Arquivo no formato shapefile do Condomínio RK}{basicmodel}{width=0.75\textwidth}

O arquivo \textit{shapefile} contém informações de todas as áreas mapeadas no condomínio, que podem ser consultadas ao exibir a tabela de atributos. A tabela de atributos do arquivo contém dados separados por linhas e colunas, onde as colunas correspondem as informações sobre cada feição mapeada e as linhas correspondem a uma feição. As colunas identificadas:

\begin{itemize}
\item Endereço: contém informações do endereço no condomínio;
\item Conjunto: contém informações do conjunto no condomínio;
\item Quadra: contém informações da quadra no condomínio;
\item Lote: contém informações do lote no condomínio;
\item Área: indica a área da feição em metros quadrados;
\item Piscina: indica se na feição existe uma piscina ou não;
\item Terrento: indica se a feição possui edificação construída ou não;
\item Agente: indica o nome do agente que mapeou o local.
\end{itemize}


\figura{4_28_QGIS_feicoes.PNG}{Tabela de atributos do arquivo shapefile do Condomínio RK}{basicmodel}{width=0.75\textwidth}


\section{Simulação versos realidade}

A simulação é uma das mais poderosas ferramentas de análise disponível para projeto e operação de sistemas. A realização de um estudo de simulação antes da implantação do sistema real é muito importante porque permite a aceleração do funcionamento do sistema no tempo, possibilita prever os quase inevitáveis acidentes que ocorrem quando da implantação de um sistema real além de poupar recursos econômicos \cite{lobao1999evoluccao}.

Simulações computacionais vão além das simples animações. Elas englobam uma vasta classe de tecnologias, do vídeo a realidade virtual, que podem ser classificadas em certas categorias gerais baseadas fundamentalmente no grau de interatividade entre o aprendiz e o computador \cite{gaddis2000learning}. Logo, a simulação computacional é uma das ferramentas que podem ser utilizadas para a aquisição, organização e construção do conhecimento e da visão sistêmica. Esse recurso favorece a educação e o treinamento das pessoas e, consequentemente, sua adaptação às rápidas mudanças de nossa sociedade. \cite{de2014simulaccao}.


\subsection{Sistemas}

O comportamento de um sistema é estudado através de um modelo de simulação. Este modelo geralmente utiliza diversos parâmetros sobre a operação do sistema. Uma vez desenvolvido e validado, o modelo pode ser usado para investigar uma grande variedade de questões sobre o sistema. Mudanças no sistema podem ser simuladas a fim de prever seu impacto no seu desempenho. A simulação pode também ser usada para estudar sistemas ainda na fase de concepção, antes que sejam efetivamente implementados. Assim, a simulação pode ser usada como uma ferramenta para predizer os efeitos de uma mudança em sistemas existentes e também como uma ferramenta de projeto para avaliar e validar o desempenho de novos sistemas \cite{miyagi2006introduccao}.

Para modelar um sistema, é necessário assimilar o conceito de sistema e de fronteira do sistema. Um sistema é definido como um grupo de objetos que estão agregados de acordo com uma relação de interdependência para atingir certos objetivos \cite{miyagi2006introduccao}. 

Um sistema é muitas vezes afetado por mudanças que ocorrem fora do sistema. Estas mudanças ocorrem, portanto, no chamado ambiente externo do sistema. Em modelagem de sistemas, é necessário definir a fronteira entre o sistema e seu ambiente. Esta definição depende da finalidade do estudo \cite{miyagi2006introduccao}.

\subsection{Modelos}

Um modelo, em engenharia, pode ser definido como uma representação de um sistema com o intuito de estudá-lo. Para a maioria dos casos, é necessário somente considerar os aspectos do sistema que afetam esse estudo. Estes aspectos são representados no modelo do sistema, e este modelo, por definição, é uma simplificação do sistema. Por outro lado, o modelo deve ser suficientemente detalhado para permitir conclusões válidas sobre o sistema real. Diferentes modelos de um mesmo sistema podem ser necessários de acordo com o objetivo do estudo. Modelos podem ser classificados como sendo matemáticos ou físicos. Um modelo matemático usa notação simbólica e relações matemáticas para representar um sistema. Um modelo de simulação é um tipo particular de modelo matemático de um sistema \cite{miyagi2006introduccao}.

\subsection{Simulações}

A simulação computacional de sistemas envolve a aquisição de conhecimento. Segundo Dibella \& Nevis, a aquisição de conhecimento está relacionada ao desenvolvimento de novos conceitos e métodos e à identificação de ideias, habilidades e relacionamentos \cite{dibella1999organizaccoes}. Ainda, segundo Nonaka \& Takeuchi, propõem que o conhecimento é adquirido por meio da interação entre o conhecimento tácito e o conhecimento explícito \cite{nonaka1997criaccao}. Eles estabelecem quatro modos diferentes de conversão do conhecimento:

\begin{itemize}
\item de conhecimento tácito em conhecimento tácito: socialização; 
\item de conhecimento tácito em conhecimento explícito: externalização; 
\item de conhecimento explícito em conhecimento explícito: combinação; 
\item de conhecimento explícito em conhecimento tácito: internalização. 
\end{itemize}

\figura{5_3_conhecimento.PNG}{Modos de conversão do conhecimento \cite{nonaka1997criaccao}}{basicmodel}{width=0.75\textwidth}

\subsubsection{Conhecimento tácito x Conhecimento explícito}

Os formatos tácito e explícito do conhecimento podem ser bem entendidos e diferenciados pela analogia que Bond \& Otterson (1998) \cite{bond1998creativity} fazem com o trabalho de um artesão, escultor de madeira. Esse escultor experiente pode escrever detalhadas regras e procedimentos e construir elaboradas ferramentas físicas, porém, consegue incorporar nestes recursos formais apenas parte de seu conhecimento, aqueles que o escultor consegue externalizar na forma explícita. Mas quando este inicia uma nova escultura, um dos mais importantes conhecimentos é a visão do todo, do resultado final de seu trabalho, que o vai guiar nos detalhes e pequenas tarefas para conseguir seu intento. Esse conhecimento é tácito, apenas precariamente externalizado na forma de um desenho ou discurso, mas pode ser razoavelmente bem captado e incorporado por um aprendiz deste artesão que junto com ele trabalha por um regular intervalo de tempo \cite{silva2004gestao}.

Já Mascitelli (2000) \cite{mascitelli2000experience} ilustra com a diferenciação das habilidades entre dois pianistas, ambos músicos. Ao descrever que dois pianistas, um aprendiz e um mestre, podem ter acesso ao mesmo conhecimento explícito (as partituras musicais), porém, o entendimento destas partituras será diferente, assim como a reação a esta leitura, em termos de interação com o teclado e ajuste ao som produzido (conhecimento tácito) \cite{silva2004gestao}.

O formato tácito, conhecimento subjetivo é um conjunto de habilidades inerentes a uma pessoa; sistema de ideias, percepção e experiência; difícil de ser formalizado, transferido ou explicado a outra pessoa \cite{silva2004gestao}.

O formato explícito, conhecimento relativamente fácil de codificar, transferir e reutilizar; formalizado em textos, gráficos, tabelas, figuras, desenhos, esquemas, diagramas, etc., facilmente organizados em bases de dados e em publicações em geral, tanto em papel quanto em formato eletrônico \cite{silva2004gestao}.


\subsubsection{Socialização}

A socialização é a conversão de parte do conhecimento tácito de uma pessoa no conhecimento tácito de outra pessoa. Esse tipo de conversão também é abordado pelas teorias ligadas à cultura organizacional e ao trabalho em grupo \cite{silva2004gestao}. Um exemplo seria compartilhar informações por reuniões online.

\subsubsection{Externalização}

A externalização é um processo no qual o conhecimento tácito se torna explícito, expresso na forma de modelos, metáforas, analogias, conceitos ou hipóteses. Esse modo de conversão do conhecimento é a chave para a criação do conhecimento, pois gera conceitos novos e explícitos a partir do conhecimento tácito \cite{gavira2003simulaccao}.


\subsubsection{Combinação}

A combinação é um processo de sistematização de conceitos em um sistema de conhecimento; ela envolve a combinação de diferentes conjuntos de conhecimento explícito. As pessoas trocam e combinam conhecimentos através de meios como documentos, reuniões, conversas, redes de comunicação computadorizadas etc. A reconfiguração das informações existentes através da classificação, do acréscimo, da combinação e da categorização do conhecimento explícito pode levar a novos conhecimentos \cite{gavira2003simulaccao}.

\subsubsection{Internalização}

Internalização é o processo de inclusão do conhecimento explícito no 
conhecimento tácito; está intimamente relacionada ao “aprender fazendo”. As experiências, incluindo aquelas adquiridas nos outros modos de conversão, são internalizadas no conhecimento tácito dos indivíduos sob a forma de modelos mentais ou conhecimento técnico \cite{gavira2003simulaccao}.

\subsubsection{Metáfora e Analogia}

A metáfora é uma forma de perceber ou entender algo intuitivamente, através da criação de uma imagem simbólica composta de outros elementos; em outras palavras, faz com que a pessoa entenda determinado problema relacionando-o com uma situação ou objeto conhecido. Através dela, pode-se entender conceitos abstratos pelo uso de imagens de elementos concretos \cite{gavira2003simulaccao}.

A analogia destaca o caráter comum a duas coisas diferentes. A associação entre coisas ou fatos através da analogia é realizada pelo pensamento racional e concentra-se nas semelhanças entre elas. O modelo, por sua vez, é normalmente criado a partir de metáforas; nele, são representados os conceitos explícitos, não devendo haver contradições. Todos os conceitos e proposições devem ser expressos em linguagem sistemática e lógica \cite{gavira2003simulaccao}.

Logo, de acordo com Gaviria 2003, podemos assumir que um modelo de simulação de um sistema é uma forma de converter conhecimento tácito em codificado, já que as atividades de reflexão, discussão, dedução, indução etc. inerentes à modelagem proporcionam o conhecimento do sistema (aquisição de conhecimento sobre o sistema modelado) \cite{gavira2003simulaccao}.


\subsection{Simulação do RK versus realidade dos dados}

A simulação computacional do RK permitiu o conhecimento através da experimentação ao analisar diversos cenários com políticas que mitigam a disseminação do COVID-19.

Foi possível observar baseado nos dados disponibilizados por período epidemiológico fornecido pelo Centro de Estudos Ambientais do Condomínio Rural Residencial RK (CEA/RK) que o modelo de simulação do COMOKIT, mesmo utilizando o gerador embutido sem a personalização dos agentes e alterações na modelagem proposta pelo COMOKIT que a simulação esteve próxima quando aplicado políticas restritivas.

O modelo que aplica a política restritiva chamada Lockdown aplicou efetivamente a política em 15 dias, ou seja, nesse período efetivamente não houve aumento de casos. Após o período de 15 dias, nenhuma política é aplicada. A disseminação do vírus aumenta expressivamente, visto que nessa população, se não houver políticas que inibem a transmissão do COVID-19, um aumento ocorrerá.

O modelo que inibe a interação de agentes com outros agentes mostrou eficiência ao reduzir os casos no condomínio e não deixou surgir novos casos. Apesar das fragilidades do modelo como não modelar outras variáveis possíveis que podem causar a contaminação entre agentes, o isolamento social foi o suficiente para eliminar a COVID-19 no condomínio.

O modelo que simula escolas e trabalhos fechados foi o mais próximo da realidade, pois em um contexto genérico foi um possível cenário real no Condomínio RK. 


