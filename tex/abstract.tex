Understanding the way the pandemic behaves is fundamental for harm reduction and control analyses. The QGIS Tool was used for the mapping and the Language GAMA and COMOKIT for the execution of agent-based simulations. Studies were carried out on two regions in the DF during a certain period of time. The first study was in the Santa Luzia region, an irregular community with imprecise data located in Cidade Estrutural in Brasília-DF and the second study was in Condomínio RK, the most regular and best established, located in Sobradinho, Brasília-DF. The objective of the studies was to show the spread of Covid-19 in these places that underwent restrictive policies, such as the closing of schools and lockdowns in order to generate simulations consistent with what actually happened in the analyzed period.