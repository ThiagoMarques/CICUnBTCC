\section{Execução}

A execução da simulação consiste em iniciar o COMOKIT e carregar o espaço de trabalho com dados das simulações. O GAMA (com COMOKIT) deve estar instalado e funcionando \cite{ComokitDoc}. A simulação foi realizada no Condomínio Rural Residencial RK, localizado em uma área rural de Brasília-DF.


\subsection{Dados espaciais}

O Condomínio Rural Residencial RK, fundado em 1992, é localizado na Região dos Lagos em Sobradinho-DF, próximo a uma extensa área de vegetação nativa, afastado do centro urbano. A sua entrada, anterior à guarita, há lotes comerciais com comerciantes parceiros do condomínio, característica que facilita o acesso da comunidade a produtos e serviços da região, incentivando a economia local \cite{CondominioRK:online}.

O Condomínio RK foi construído no Rancho Karina, com 148.188,85 hectares de extensão, dividido em 2.080 lotes residenciais e 41 lotes comerciais, com cerca de 1.900 imóveis construídos e 8.000 moradores \cite{CondominioRK:online}.

A fim de administrar a saúde ambiental, considerando os aspectos geográficos em que o condomínio se insere, foi criado em 2013, em Assembleia Geral Extraordinária, o Centro de Estudos Ambientais do Condomínio Rural Residencial RK (CEA/RK), ligado à administração do condomínio, com o objetivo de controlar e reduzir riscos à saúde da comunidade \cite{CondominioRK:online}.

Faz-se, assim, como proposta para o CEA/RK integrar diferentes áreas do
conhecimento – Ciências Ambientais, Medicina Veterinária e Saúde Coletiva –
utilizando-se da coleta e análise de dados, reconhecimento das demandas da
comunidade e aplicação de ações para promoção da Saúde Única. \cite{miranda2020acompanhamento}

O Centro de Estudos Ambientais do Condomínio Rural Residencial RK (CEA/RK) se aproxima ao serviço de vigilância, pois a medida que informações apreendidas a campo a partir do contato com a comunidade levam às ações tomadas – “informação para ação” \cite{wunsch1993sistema}.

Na prática, o papel da vigilância, então, consiste em atuar localmente acessando riscos e implementando ações de proteção e vigilância da saúde de populações, à medida em que entende a saúde de uma comunidade como dependente de um equilíbrio do todo, como um único organismo \cite{romao2019aspectos}.

Logo, o CEA/RK assume então o papel de vigilância, fiscalização e promoção da saúde à medida em que compõe rede de corresponsabilidade com Estado, suprindo as necessidades da comunidade por vezes não atendidas pela esfera pública \cite{miranda2020acompanhamento}.

\subsection{Dados Demográficos}

Os dados disponibilizados pelo CEA/RK é fruto de um estudo interno da população do condomínio. Abaixo, um gráfico sobre o perfil etário dos moradores do Condomínio RK:

\figura{4_1_perfil_etário_RK}{Perfil etário dos moradores do Condomínio RK}{basicmodel}{width=0.75\textwidth}

A população absoluta informada foi de 8000 moradores, com idade média de 35 anos e idade máxima de 93 anos. O condomínio possui uma média de quatro residentes por casa, totalizando em média 1900 habitações. O condomínio possui comércio local, com aproximadamente 41 imóveis comerciais na entrada do condomínio. A região possui uma densidade demográfica de 5,4 habitantes por quilômetro quadrado e uma área total de 1.481 km² \cite{}.

\section{Passos para como executar}

Os arquivos do COMOKIT são disponibilizados na plataforma de hospedagem GitHub. O GitHub é uma plataforma de hospedagem de código para controle de versão e colaboração \cite{githubdocumentation:online}.

O repositório do COMOKIT no Github possui arquivos de um modelo GAMA na avaliação e comparação de políticas de intervenção contra a pandemia de CoVid-19. Esses modelos ilustram como a modelagem baseada em agente pode ser usada para avaliar os impactos de várias políticas de contenção em estudos de caso definidos por seus limites geográficos \cite{GithubCOMOKIT:online}.

Ainda, de acordo com a documentação do COMOKIT, é necessário realizar o download do modelo no GitHub e extrair os arquivos no formato ZIP em algum lugar do seu computador. Por fim, importar os arquivos do COMOKIT no Gama e inseri-las no espaço para modelos do projeto \cite{ComokitDoc}.

\figura{4_2_gama.PNG}{Inicialização do aplicativo GAMA}{basicmodel}{width=0.75\textwidth}

A importação dos arquivos do COMOKIT deve ser realizada na pasta de modelo do GAMA. Ao extrair os arquivos baixados do repositório COMOKIT no GitHub, basta indicar o caminho do local extraído e importar como um projeto GAMA.

\figura{4_3_import_GAMA.PNG}{Importação de um projeto GAMA}{basicmodel}{width=0.75\textwidth}

\subsubsection{Configuração dos arquivos}

Após a importação, a pasta do COMOKIT estará disponível para abertura. A estrutura de pastas do COMOKIT:

\begin{itemize}
\item \textit{beta\_output}: pasta que armazena um arquivo CSV da calibração do valor da taxa de transmissão (β);
\item \textit{Datasets}: pasta com arquivos demográficos e espaciais;
\item \textit{Experiments}: pasta com todos os experimentos do COMOKIT.;
\item \textit{Model}: pasta que contém as entidades e modelos da simulação;
\item \textit{Parameters}: pasta que contém parâmetros epidemiológicos da simulação;
\item \textit{Utilities}: pasta que contém arquivos utilitários como os arquivos GAMA que geram pontos e edificações nas simulações, um arquivo GAMA responsável por calibrar o experimento.
\end{itemize}

As informações demográficas e espaciais do condomínio RK foram inseridas na subpasta RK em Datasets e estão disponíveis no repositório https://github.com/ThiagoMarques/comokitcovidunb. O COMOKIT necessita de um arquivo no formato \textit{shapefile} para compor as limitações geográficas. Para inserção de dados espaciais no COMOKIT, é necessário inserir obrigatoriamente um arquivo com a nomenclatura "\textit{boundary.shp}", indicando que este arquivo contém as limitações de estudo do caso do condomínio RK.

\subsubsection{Configuração dos parâmetros}

Ao realizar o download dos arquivos COMOKIT, alguns arquivos de exemplos estão contidos na pasta \textit{Datasets}. É necessário indicar no arquivos de parâmetros da simulação o nome do estudo de caso padrão e a pasta que está contido. 

Portanto, no arquivo \textit{Parameters.gaml} na pasta \textit{Model}, é necessário alterar a \textit{string} DEFAULT\_CASE\_STUDY\_FOLDER\_NAME e indicar o nome da pasta criada chamada "RK", conforme a imagem abaixo:

\figura{4_4_parameters_RK.PNG}{Parâmetros da simulação}{basicmodel}{width=0.75\textwidth}


\subsubsection{Gerar edificações e pontos}

O COMOKIT fornece um modelo que solicita servidores OSM com 
 relação a área definida em \textit{boundary.shp} e constrói automaticamente o arquivo. 
É necessário abrir o arquivo "\textit{Generate Buildings from Points.gaml}" localizado na pasta Utilites e indicar a pasta com o arquivo que delimita a área geográfica da simulação (\textit{boundary.shp}).

\figura{4_5_generate_buildings.PNG}{Parâmetros para gerar edificações e pontos na simulação}{basicmodel}{width=0.75\textwidth}

Ao iniciar a execução do arquivo "\textit{Generate Buildings from Points.gaml}", um mapa é apresentado com informações geográficas obtidas a partir do arquivo "\textit{boundary.shp}":

\begin{itemize}
\item 2919 blocos e pontos reconhecidos;
\item \textit{background} disponível.
\end{itemize}


\figura{4_6_generate_buildings_map.PNG}{Experimento pronto para gerar dados de edificações no GAMA}{basicmodel}{width=0.75\textwidth}

Após realizar a execução do experimento "\textit{Generate Buildings from Points.gaml}", é inserido automaticamente o arquivo "\textit{buildings.shp}" e imagens de satélites na pasta RK, localizada em \textit{Datasets}.


\section{Apresentação das execuções}


\section{Coleta e organização dos dados}

\section{Execução da Santa Luzia e dados coletados}

\section{Execução do RK e dados coletados}



