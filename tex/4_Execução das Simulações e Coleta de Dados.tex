\section{Execução}

A execução da simulação consiste em iniciar o COMOKIT e carregar o espaço de trabalho com dados das simulações. O GAMA (com COMOKIT) deve estar instalado e funcionando \cite{ComokitDoc}. A simulação foi realizada no Condomínio Rural Residencial RK, localizado em uma área rural de Brasília-DF.


\subsection{Dados espaciais}

O Condomínio Rural Residencial RK, fundado em 1992, é localizado na Região dos Lagos em Sobradinho-DF, próximo a uma extensa área de vegetação nativa, afastado do centro urbano. A sua entrada, anterior à guarita, há lotes comerciais com comerciantes parceiros do condomínio, característica que facilita o acesso da comunidade a produtos e serviços da região, incentivando a economia local \cite{CondominioRK:online}.

O Condomínio RK foi construído no Rancho Karina, com 148.188,85 hectares de extensão, dividido em 2.080 lotes residenciais e 41 lotes comerciais, com cerca de 1.900 imóveis construídos e 8.000 moradores \cite{CondominioRK:online}.

A fim de administrar a saúde ambiental, considerando os aspectos geográficos em que o condomínio se insere, foi criado em 2013, em Assembleia Geral Extraordinária, o Centro de Estudos Ambientais do Condomínio Rural Residencial RK (CEA/RK), ligado à administração do condomínio, com o objetivo de controlar e reduzir riscos à saúde da comunidade \cite{CondominioRK:online}.

Faz-se, assim, como proposta para o CEA/RK integrar diferentes áreas do
conhecimento – Ciências Ambientais, Medicina Veterinária e Saúde Coletiva –
utilizando-se da coleta e análise de dados, reconhecimento das demandas da
comunidade e aplicação de ações para promoção da Saúde Única. \cite{miranda2020acompanhamento}

O Centro de Estudos Ambientais do Condomínio Rural Residencial RK (CEA/RK) se aproxima ao serviço de vigilância, pois a medida que informações apreendidas a campo a partir do contato com a comunidade levam às ações tomadas – “informação para ação” \cite{wunsch1993sistema}.

Na prática, o papel da vigilância, então, consiste em atuar localmente acessando riscos e implementando ações de proteção e vigilância da saúde de populações, à medida em que entende a saúde de uma comunidade como dependente de um equilíbrio do todo, como um único organismo \cite{romao2019aspectos}.

Logo, o CEA/RK assume então o papel de vigilância, fiscalização e promoção da saúde à medida em que compõe rede de corresponsabilidade com Estado, suprindo as necessidades da comunidade por vezes não atendidas pela esfera pública \cite{miranda2020acompanhamento}.

\subsection{Dados Demográficos}

Os dados disponibilizados pelo CEA/RK é fruto de um estudo interno da população do condomínio. Abaixo, um gráfico sobre o perfil etário dos moradores do Condomínio RK:

\figura{4_1_perfil_etário_RK}{Perfil etário dos moradores do Condomínio RK}{basicmodel}{width=0.75\textwidth}

A população absoluta informada foi de 8000 moradores, com idade média de 35 anos e idade máxima de 93 anos. O condomínio possui uma média de quatro residentes por casa, totalizando em média 1900 habitações. O condomínio possui comércio local, com aproximadamente 41 imóveis comerciais na entrada do condomínio. A região possui uma densidade demográfica de 5,4 habitantes por quilômetro quadrado e uma área total de 1.481 km² \cite{}.






\section{Passos para como executar}
\sections{Apresentação das execuções}
\section{Coleta e organização dos dados}

\section{Execução da Santa Luzia e dados coletados}

\section{Execução do RK e dados coletados}



