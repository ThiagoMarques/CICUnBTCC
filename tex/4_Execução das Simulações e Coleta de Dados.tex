\section{Execução}

A execução da simulação consiste em iniciar o COMOKIT e carregar o espaço de trabalho com dados das simulações. O GAMA (com COMOKIT) deve estar instalado e funcionando \cite{ComokitDoc}. A simulação foi realizada no Condomínio Rural Residencial RK, localizado em uma área rural de Brasília-DF.


\subsection{Dados espaciais}

O Condomínio Rural Residencial RK, fundado em 1992, é localizado na Região dos Lagos em Sobradinho-DF, próximo a uma extensa área de vegetação nativa, afastado do centro urbano. A sua entrada, anterior à guarita, há lotes comerciais com comerciantes parceiros do condomínio, característica que facilita o acesso da comunidade a produtos e serviços da região, incentivando a economia local \cite{CondominioRK:online}.

O Condomínio RK foi construído no Rancho Karina, com 148.188,85 hectares de extensão, dividido em 2.080 lotes residenciais e 41 lotes comerciais, com cerca de 1.900 imóveis construídos e 8.000 moradores \cite{CondominioRK:online}.

A fim de administrar a saúde ambiental, considerando os aspectos geográficos em que o condomínio se insere, foi criado em 2013, em Assembleia Geral Extraordinária, o Centro de Estudos Ambientais do Condomínio Rural Residencial RK (CEA/RK), ligado à administração do condomínio, com o objetivo de controlar e reduzir riscos à saúde da comunidade \cite{CondominioRK:online}.

Faz-se, assim, como proposta para o CEA/RK integrar diferentes áreas do conhecimento – Ciências Ambientais, Medicina Veterinária e Saúde Coletiva – utilizando-se da coleta e análise de dados, reconhecimento das demandas da comunidade e aplicação de ações para promoção da Saúde Única. \cite{miranda2020acompanhamento}

O Centro de Estudos Ambientais do Condomínio Rural Residencial RK (CEA/RK) se aproxima ao serviço de vigilância, pois a medida que informações apreendidas a campo a partir do contato com a comunidade levam às ações tomadas – “informação para ação” \cite{wunsch1993sistema}.

Na prática, o papel da vigilância, então, consiste em atuar localmente acessando riscos e implementando ações de proteção e vigilância da saúde de populações, à medida em que entende a saúde de uma comunidade como dependente de um equilíbrio do todo, como um único organismo \cite{romao2019aspectos}.

Logo, o CEA/RK assume então o papel de vigilância, fiscalização e promoção da saúde à medida em que compõe rede de corresponsabilidade com Estado, suprindo as necessidades da comunidade por vezes não atendidas pela esfera pública \cite{miranda2020acompanhamento}.

\subsection{Dados Demográficos}

Os dados disponibilizados pelo CEA/RK é fruto de um estudo interno da população do condomínio. Abaixo, um gráfico sobre o perfil etário dos moradores do Condomínio RK:

\figura{4_1_perfil_etário_RK}{Perfil etário dos moradores do Condomínio RK}{basicmodel}{width=0.75\textwidth}

De acordo com dados disponibilizados pelo Centro de Estudos Ambientais do Condomínio Rural Residencial RK (CEA/RK), a população absoluta informada foi de 8000 moradores, com idade média de 35 anos e idade máxima de 93 anos. O condomínio possui uma média de quatro residentes por casa, totalizando em média 1900 habitações. O condomínio possui comércio local, com aproximadamente 41 imóveis comerciais na entrada do condomínio. A região possui uma densidade demográfica de 5,4 habitantes por quilômetro quadrado e uma área total de 1.481 km².

\section{Passos para como executar}

Os arquivos do COMOKIT são disponibilizados na plataforma de hospedagem GitHub. O GitHub é uma plataforma de hospedagem de código para controle de versão e colaboração \cite{githubdocumentation:online}.

O repositório do COMOKIT no Github possui arquivos de um modelo GAMA na avaliação e comparação de políticas de intervenção contra a pandemia de CoVid-19. Esses modelos ilustram como a modelagem baseada em agente pode ser usada para avaliar os impactos de várias políticas de contenção em estudos de caso definidos por seus limites geográficos \cite{GithubCOMOKIT:online}.

Ainda, de acordo com a documentação do COMOKIT, é necessário realizar o download do modelo no GitHub e extrair os arquivos no formato ZIP em algum lugar do seu computador. Por fim, importar os arquivos do COMOKIT no Gama e inseri-las no espaço para modelos do projeto \cite{ComokitDoc}.

\figura{4_2_gama.PNG}{Inicialização do aplicativo GAMA}{basicmodel}{width=0.75\textwidth}

\figura{4_3_import_GAMA.PNG}{Importação de um projeto GAMA}{basicmodel}{width=0.75\textwidth}

A importação dos arquivos do COMOKIT deve ser realizada na pasta de modelo do GAMA. Ao extrair os arquivos baixados do repositório COMOKIT no GitHub, é necessário indicar o caminho do local extraído e importar como um projeto GAMA.

\subsubsection{Configuração dos arquivos}

Após a importação, a pasta do COMOKIT estará disponível para abertura. Abaixo, a imagem com a estrutura de pastas do COMOKIT:

\begin{itemize}
\item \textit{beta\_output}: pasta que armazena um arquivo CSV da calibração do valor da taxa de transmissão ($\beta$);
\item \textit{Datasets}: pasta com arquivos demográficos e espaciais;
\item \textit{Experiments}: pasta com todos os experimentos do COMOKIT.;
\item \textit{Model}: pasta que contém as entidades e modelos da simulação;
\item \textit{Parameters}: pasta que contém parâmetros epidemiológicos da simulação;
\item \textit{Utilities}: pasta que contém arquivos utilitários como os arquivos GAMA que geram pontos e edificações nas simulações, um arquivo GAMA responsável por calibrar o experimento.
\end{itemize}

As informações demográficas e espaciais do condomínio RK foram inseridas na subpasta RK em \textit{data/datasets} e estão disponíveis no repositório:
\begin{itemize}
\item https://github.com/ThiagoMarques/CICUnBTCC
\end{itemize}

O COMOKIT necessita de um arquivo no formato \textit{shapefile} para compor as limitações geográficas. Para inserção de dados espaciais no COMOKIT, é necessário inserir obrigatoriamente um arquivo com a nomenclatura "\textit{boundary.shp}", indicando que este arquivo contém as limitações de estudo do caso do condomínio RK.

\subsubsection{Configuração dos parâmetros}

Ao realizar o download dos arquivos COMOKIT, alguns arquivos de exemplos estão contidos na pasta \textit{Datasets}. É necessário indicar no arquivos de parâmetros da simulação o nome do estudo de caso padrão e a pasta que está contido. 

Portanto, no arquivo \textit{Parameters.gaml} na pasta \textit{Model}, é necessário alterar a \textit{string} DEFAULT\_CASE\_STUDY\_FOLDER\_NAME e indicar o nome da pasta criada chamada "RK", conforme a imagem abaixo:

\figura{4_4_parameters_RK.PNG}{Parâmetros da simulação}{basicmodel}{width=0.75\textwidth}


\subsubsection{Gerar edificações e pontos}

O COMOKIT fornece um modelo que solicita servidores OSM (Open Street Map) com relação a área definida em \textit{boundary.shp} e constrói automaticamente o arquivo. 
É necessário abrir o arquivo "\textit{Generate Buildings from Points.gaml}"  localizado na pasta Utilites e indicar a pasta com o arquivo que delimita a área geográfica da simulação (\textit{boundary.shp}).

\figura{4_5_generate_buildings.PNG}{Parâmetros para gerar edificações e pontos na simulação}{basicmodel}{width=0.75\textwidth}

Ao iniciar a execução do arquivo "\textit{Generate Buildings from  points.gaml}", um mapa é apresentado com informações geográficas obtidas a partir do arquivo "\textit{boundary.shp}":

\begin{itemize}
\item 2919 blocos e pontos reconhecidos;
\item \textit{background} disponível.
\end{itemize}


\figura{4_6_generate_buildings_map.PNG}{Experimento pronto para gerar dados de edificações no GAMA}{basicmodel}{width=0.75\textwidth}

Após realizar a execução do experimento "\textit{Generate Buildings from Points.gaml}", é inserido automaticamente o arquivo "\textit{buildings.shp}" e imagens de satélites na pasta RK, localizada em \textit{Datasets}.

Para aproximação da realidade, o Centro de Estudos Ambientais do Condomínio Rural Residencial RK (CEA/RK) disponibilizou o mapa do condomínio no formato \textit{shapefile} com informações das edificações do condomínio:

\figura{4_27_QGIS_RK.PNG}{Imagem do condomínio RK no software QGIS}{basicmodel}{width=0.75\textwidth}

No software QGIS, foi inserido na tabela de atributos as informações sobre os tipos de edificação em cada ponto mapeado para identificação de residências e comércios. De acordo com a documentação do QGIS, cada linha da tabela representa uma feição, podendo conter uma geometria ou não, e cada coluna contém uma informação específica sobre a feição. As feições podem ser pesquisadas, selecionadas, movidas ou editadas \cite{DocQGIS:online}.

\figura{4_28_QGIS_feicoes.PNG}{Feições do condomínio RK no software QGIS}{basicmodel}{width=0.75\textwidth}

Após edição da tabela de atributos no QGIS, o arquivo \textit{shapefile} chamado \textit{boundary.shp} foi exportado para a pasta RK contida no COMOKIT e utilizada para gerar as simulações.



\section{Apresentação das execuções}

O primeiro diagnóstico de Covid-19 no Distrito Federal foi no dia 05 de março de 2020 \cite{SESCovid19S7:online}. O caso tratava-se de uma mulher, de 53 anos, que viajou pela Inglaterra e Suíça \cite{Brasilco67:online}. No dia 14 de março de 2022, foi publicado o decreto Nº 40.520 publicado no Diário Oficial da União que aplicava de acordo com a imprensa \cite{CorreioBraziliense032020:online} o primeiro lockdown no Distrito Federal.

O decreto Nº 40520 suspendeu pelo prazo de quinze dias eventos, atividades coletivas de cinema e teatro, atividades educacionais em todas as escolas, universidades, faculdades das redes públicas e privadas, academias de esportes de todas as modalidades \cite{Brasilco67:online}. O decreto Nº 40522 incluiu a suspensão aos museus \cite{Decreto441:online} e o decreto Nº 40529 incluiu a suspensão ao zoológico, parques ecológicos, recreativos, urbanos, vivenciais e afins e boates e casas noturnas, atendimento ao público em shoppings centers, feiras populares e clubes recreativos \cite{Decreto412:online}.

No dia 18 de março, após a publicação do decreto Nº 40.520 e incisos acrescentados ao decreto, o cenário no DF de acordo com o jornal Correio Braziliense indicava que na região do Distrito Federal havia 36 casos confirmados do SARS-CoV2 \cite{CorreioBraziliense032020:online}.

Os dados divulgados do condomínio RK pelo Centro de Estudos Ambientais do Condomínio Rural Residencial RK (CEA/RK) mostram que no período de 15 de março de 2020 a 21 de março de 2020 existia 1 caso registrado no condomínio.

\figura{4_8_CEA_RK_1.jpeg}{Número total de novos casos no condomínio RK por semana epidemiológica de início de sintomas}{basicmodel}{width=0.75\textwidth}

\subsection{Simulação no COMOKIT}

Para visualização da simulação, é apresentado um painel com os parâmetros configurados, um gráfico com a evolução epidemiológica de casos de COVID na simulação e um mapa do condomínio interativo com os agentes separados por cores.

\figura{4_11_lockdown_ready.PNG}{Modelo Lockdown no COMOKIT pronto para execução de simulação}{basicmodel}{width=0.75\textwidth}

Cada simulação importa pelo menos dois arquivos abaixo para realizar a execução:
\begin{enumerate}
\item \textit{Global.gaml}
\item \textit{Abstract Experiment.gaml}
\end{enumerate}

\subsubsection{Global.gaml}

O arquivo \textit{Global.gaml} contém declarações globais de ações e atributos, usados principalmente para inicializar o modelo em experimentos. Esse modelo importa entidades que de acordo com a documentação do COMOKIT, as entidades representam os habitantes individuais com suas características de sexo, situação de emprego, idade e situação epidemiológica e entidades espaciais que são edificações onde os agentes individuais realizam atividades, que dependem do tipo de edifício definido \cite{ComokitDoc}.

\figura{4_29_global_gaml.PNG}{Importações no arquivo Global.gaml}{basicmodel}{width=0.75\textwidth}

Este arquivo é o responsável por criar as edificações e atividades da simulação e inserir os parâmetros epidemiológicos. No condomínio RK, como não existe uma edificação do tipo Hospital, o COMOKIT constrói automaticamente uma edificação especial fora da região delimitada no arquivo \textit{boundary.shp} para simular um hospital. Caso exista um hospital inserido no arquivo \textit{shapefile}, o código abaixo deve ser removido:

\figura{4_30_Global_gaml_hospital.PNG}{Simulação de hospital no arquivo Global.gaml}{basicmodel}{width=0.75\textwidth}

\subsubsection{Abstract Experiment.gaml}

O arquivo \textit{Abstract Experiment.gaml} insere variáveis que são responsáveis pela interface gráfica da simulação, como cores, gráficos e imagens de fundo da simulação. Alguns parâmetros importantes para identificação em tempo real dos casos da simulação são feitas neste arquivo, como a identificação de cores dos agentes com situação epidemiológica. 

As cores abaixo representam o status epidemiológico dos agentes, representados no mapa por um quadrado:

\begin{itemize}
\item verde: representa um agente saudável, que não possui status recuperado, suscetível ou infectado;
\item azul: representa um agente recuperado;
\item amarelo: representa um agente suscetível;
\item laranja: representa um agente infectado;
\item preta: representa um agente morto.
\end{itemize}


\subsection{Lockdown}

O COMOKIT fornece um modelo chamado Lockdown que simula a política de bloqueio total aplicada a um determinado número de dias. O modelo simula o bloqueio total de atividades, que é aplicada a todos os indivíduos. Após a finalização do período inicial de dias de lockdown configurado, nenhuma política é aplicada na simulação. 

\figura{4_31_lockdown_basic.PNG}{Modelo Lockdown no COMOKIT}{basicmodel}{width=0.75\textwidth}

\subsubsection{Importações}

O modelo utiliza os arquivos \textit{Global.gaml }para instanciar agentes individuais e agentes espaciais e o arquivo \textit{Abstract Experiment.gaml } para exibição de interface gráfica.

\subsubsection{Definições globais}

O número de 15 dias foi configurado para execução da simulação na variável \textit{num\_days}:

\figura{4_10_model_lockdown.PNG}{Configuração do modelo Lockdown no COMOKIT}{basicmodel}{width=0.75\textwidth}

A política aplicada foi nomeada como create\_lockdown\_policy, herdada do arquivo Authority.gaml. A política percorre uma lista de atividades chamada ActivitiesListingPolicy, onde todas as atividades dos agentes são configuradas para o valor falso:

\figura{4_32_ActivitiesListingPolicy.PNG}{Aplicação da política de Lockdown na lista de atividades dos agentes}{basicmodel}{width=0.75\textwidth}

\subsubsection{Interface gráfica}

Para exibição do gráfico e informações da simulação, foi configurado o tipo default\_display para exibição dos agentes no mapa e para exibição do gráfico, foi configurado o tipo states\_evolution\_chart:

\figura{4_33_lockdown_grafico.PNG}{Configuração da interface gráfica para execução do modelo Lockdown}{basicmodel}{width=0.75\textwidth}


\subsubsection{Parâmetros globais}

Para realizar a simulação do lockdown no COMOKIT, foi ajustado os parâmetros iniciais que define a data inicial da simulação, ajustada para o dia 14 de março de 2020 e o número inicial de infectados, ajustado para 1 caso no condomínio.

\figura{4_7_parameters_lockdown.PNG}{Parâmetros para realizar a simulação do lockdown no condomínio RK}{basicmodel}{width=0.75\textwidth}

\subsubsection{Resultados}

Para realizar a simulação do lockdown, foi ajustado os parâmetros iniciais que define a data inicial da simulação, ajustada para o dia 14 de março de 2020 e o número inicial de infectados, ajustado para 1 caso no condomínio e finalizaram no dia 15 de outubro de 2020, totalizando 7 meses, 214 dias e aproximadamente 5136 horas.

\figura{4_24_lockdown_1.PNG}{Resultado da simulação do modelo lockdown no condomínio RK}{basicmodel}{width=0.75\textwidth}


\subsection{No Activities with Meeting}

O COMOKIT fornece um modelo chamado No Activities with Meeting que simula uma política restritiva que impede os de interagir com outros agentes. As atividades como trabalhar, estudar, ir à escola, comer, lazer ou esporte são proibidas.

\figura{4_34_NoActivitiesMeeting.PNG}{Modelo No Activities with Meeting COMOKIT}{basicmodel}{width=0.75\textwidth}

\subsubsection{Importações}

O modelo utiliza os arquivos \textit{Global.gaml} para instanciar agentes individuais e agentes espaciais e o arquivo \textit{Abstract Experiment.gaml} para exibição de interface gráfica.

\subsubsection{Definições globais}

A política aplicada foi nomeada como create\_no\_meeting\_policy, herdada do arquivo \textit{Authority.gaml}. A política percorre uma lista de atividades chamada \textit{ActivitiesListingPolicy}. As atividades definidas como uma atividade com reunião são definidas nos parâmetros globais do COMOKIT. Essa variável global chamada de meeting\_relaxing\_act contém as atividades dos agentes inseridas na lista, que serão proibidas ao escolher essa política. As definições dessa variável estão nas configurações globais de parâmetros (\textit{Parameters.gaml}):

\figura{4_36_meeting_relaxing.PNG}{Parâmetros para atividades proibidas ao escolher a política No activity with meeting}{basicmodel}{width=0.75\textwidth}

Abaixo, o retorno configurado para falso de todas as atividades dos agentes.

\figura{4_35_CreateNoMeetingPolicy.PNG}{Aplicação da política No activity with meeting na lista de atividades dos agentes}{basicmodel}{width=0.75\textwidth}


\subsubsection{Interface gráfica}

Para exibição do gráfico e informações da simulação, foi configurado o tipo default\_display para exibição dos agentes no mapa e para exibição do gráfico, foi configurado o tipo states\_evolution\_chart:

\figura{4_32_ActivitiesListingPolicy_parameters.PNG}{Configuração da interface gráfica para execução da simulação do modelo No activity with meeting}{basicmodel}{width=0.75\textwidth}


\subsubsection{Parâmetros globais}

Para realizar a simulação do modelo No activity with meeting no COMOKIT, foi ajustado os parâmetros iniciais que define a data inicial da simulação, ajustada para o dia 14 de março de 2020 e o número inicial de infectados, ajustado para 1 caso no condomínio.

\figura{4_7_parameters_lockdown.PNG}{Parâmetros para realizar a simulação do modelo No activity with meeting no condomínio RK}{basicmodel}{width=0.75\textwidth}

\subsubsection{Resultados}

Para realizar a simulação da política restritiva que impede os agentes de praticar esportes e demais atividades, foi ajustado os parâmetros iniciais que define a data inicial da simulação, ajustada para o dia 14 de março de 2020 e o número inicial de infectados, ajustado para 1 caso no condomínio e finalizaram no dia 15 de outubro de 2020, totalizando 7 meses, 214 dias e aproximadamente 5136 horas.

\figura{4_25_no_active_1.PNG}{Dados da simulação após execução do modelo No activity with meeting}{basicmodel}{width=0.75\textwidth}


\subsection{Escolas e trabalhos fechados}

O COMOKIT fornece um modelo chamado School and Workplace Closures que simula uma política restritiva que impede os agentes de se deslocarem ao trabalho, escolas e universidades.

\figura{4_37_SchoolWorkplaceClosures_1.PNG}{Modelo School and Workplace Closure}{basicmodel}{width=0.75\textwidth}

\subsubsection{Importações}

O modelo utiliza os arquivos Global.gaml para instanciar agentes individuais e agentes espaciais e o arquivo Abstract Experiment.gaml para exibição de interface gráfica.

\subsubsection{Definições globais}

A política aplicada foi nomeada como create\_school\_work\_allowance\_policy, herdada do arquivo Authority.gaml. A política percorre uma lista de atividades chamada ActivitiesListingPolicy e marca como falso as atividades dos agentes de se deslocarem a locais de estudo e trabalho.

\figura{4_38_CreateSchoolWorkAllowancePolicy.PNG}{Aplicação da política create\_school\_work\_allowance\_policy na lista de atividades dos agentes}{basicmodel}{width=0.75\textwidth}


\subsubsection{Interface gráfica}

Para exibição do gráfico e informações da simulação, foi configurado o tipo default\_display para exibição dos agentes no mapa e para exibição do gráfico, foi configurado o tipo states\_evolution\_chart:

\figura{4_37_SchoolWorkplaceClosures_parameters.PNG}{Configuração da interface gráfica para execução da simulação do modelo School and Workplace Closures}{basicmodel}{width=0.75\textwidth}


\subsubsection{Parâmetros globais}

Para realizar a simulação do modelo School and Workplace Closures no COMOKIT, foi ajustado os parâmetros iniciais que define a data inicial da simulação, ajustada para o dia 14 de março de 2020 e o número inicial de infectados, ajustado para 1 caso no condomínio.

\figura{4_7_parameters_lockdown.PNG}{Parâmetros para realizar a simulação do modelo School and Workplace Closures no condomínio RK}{basicmodel}{width=0.75\textwidth}

\subsubsection{Resultados}

Para realizar a simulação de escolas e trabalhos fechados, foi ajustado os parâmetros iniciais que define a data inicial da simulação, ajustada para o dia 14 de março de 2020 e o número inicial de infectados, ajustado para 1 caso no condomínio e finalizaram no dia 15 de outubro de 2020, totalizando 7 meses, 214 dias e aproximadamente 5136 horas.

\figura{4_26_escolas_trabalhos_fechados_1.PNG}{Dados da simulação após execução do modelo School and Workplace Closures}{basicmodel}{width=0.75\textwidth}


\section{Coleta e organização dos dados}

Para execução de todas as simulações, foi utilizado o gerador embutido do COMOKIT. O número de hospitais na simulação é de uma unidade. Não há um arquivo com dados da população personalizado, os dados são gerados de acordo com as edificações e parâmetros definidos no COMOKIT.

Os locais de escola e trabalho são simulados no mapa e escolhida de forma randômica pelo COMOKIT:

\figura{4_22_escolas_trabalhos_mapa_escola_trabalho.PNG}{Exemplo de local de escola, universidades e trabalhos no mapa.}{basicmodel}{width=0.75\textwidth}

Os parâmetros da simulação foram mantidas conforme os valores fornecidos por padrão do COMOKIT que podem ser consultados em https://comokit.org/docs/parameterize.

Foi realizado um ajuste na variável basic\_viral\_release que configura a liberação ao longo do tempo da carga viral no ambiente por um indivíduo infectado, valor ajustado para 4.0.

Outro ajuste realizado na variável basic\_viral\_decrease, responsável pela diminuição ao longo do tempo da carga viral no ambiente, ajustada para o valor de 0.5.

\begin{itemize}
\item basic\_viral\_release: 4.0;
\item basic\_viral\_decrease: 0.5;
\end{itemize}

As configurações personalizadas abaixo para aproximação da realidade foram feitas com base nos dados divulgados pelo Centro de Estudos Ambientais do Condomínio Rural Residencial RK (CEA/RK) e de aproximações informada pelo IPEA (Instituto de Pesquisa Econômica Aplicada) \cite{amorim_carga_2009}.

Variáveis personalizadas para geração de dados GAML:

\begin{itemize}
\item number\_children\_mean: 2.0;
\item number\_children\_max: 3;
\item school\_age: 3;
\item active\_age: 16;
\item max\_age: 93;
\end{itemize}


Variáveis personalizadas para geração da agenda sintética:

\begin{itemize}
\item work\_hours\_begin\_min: 6;
\item work\_hours\_begin\_max: 9;
\item work\_hours\_end\_min: 13; 
\item work\_hours\_end\_max: 20;
\item school\_hours\_begin\_min: 6;
\item school\_hours\_begin\_max : 9;
\item school\_hours\_end\_min: 12;
\item school\_hours\_end\_max: 19;
\item lunch\_hours\_min: 11;
\item lunch\_hours\_max: 14;
\end{itemize}


\section{Execução do RK e dados coletados}

\subsection{Lockdown}

A simulação do lockdown aplicou a política restritiva de 15 dias resultou na infecção de 3 agentes:

\figura{4_12_lockdown_end.PNG}{Execução da simulação lockdown}{basicmodel}{width=0.75\textwidth}

Após 15 dias de execução, percebemos um aumento do número de casos, motivado pela não aplicação de política restritiva:

\figura{4_24_lockdown_1.PNG}{Execução da simulação lockdown}{basicmodel}{width=0.75\textwidth}

De acordo com o gráfico da simulação, há um declínio no número de casos após aproximadamente 1500 horas ou 62 dias. O pico de casos neste período está abaixo de 15 casos infectados.

Após aproximadamente 2000 horas de execução ou 84 dias, os casos ficaram abaixo de 5 infectados e deste ponto em diante, há um aumento expressivo no número de casos chegando a mais de 100 infectados no condomínio.

\subsection{Escolas, trabalhos e atividades de lazer fechados}

A execução da simulação que restringe o acesso as escolas, universidades e trabalhos e atividades de lazer foi iniciada com número de infectados inicial de 1 caso no condomínio. Foi observado que o número de infectados atingiu o número de 4 infectados após 20 dias. Os casos ficaram concentrados a esquerda do mapa, indicando que o contágio provavelmente aconteceu pela proximidade das pessoas na mesma área:

\figura{4_25_no_active_1.PNG}{Execução da simulação que restringe o acesso dos agentes a escola, universidades, trabalhos e atividades de lazer}{basicmodel}{width=0.75\textwidth}

Após 20 dias, não houve registro de casos na simulação.

\subsection{Escolas e trabalhos fechados}

A execução da simulação que restringe o acesso as escolas, universidades e trabalhos foi iniciada com número de infectados de 1 caso no condomínio. 

\figura{4_26_escolas_trabalhos_fechados_1.PNG}{Execução da simulação que restringe o acesso dos agentes a escola, universidades, trabalhos}{basicmodel}{width=0.75\textwidth}


Os casos anteriores a 145 dias ficaram abaixo de 20 casos. Após esse período, os casos cresceram consideravelmente, chegando ao pico de 54 casos.



