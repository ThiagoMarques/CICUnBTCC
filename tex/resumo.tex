
Entender a forma que a pandemia se comporta é fundamental para análises de controle e redução de danos. Tendo isso em mente, foi utilizada a Ferramenta QGIS para o mapeamento e a Linguagem GAMA para a execução de simulação baseada em agentes. Dessa forma foram feitos estudos sobre duas regiões no DF, durante um determinado espaço de tempo. Um deles sobre a região da Santa Luzia que é uma comunidade irregular com certa uma imprecisão nos dados localizada na Estrutural e outro no Condomínio RK regular e melhor estabelecido localizado em Sobradinho. O objetivo nos estudos realizados foi mostrar a disseminação do Covid-19 nesses locais que passaram por políticas restritivas como fechamento de escolas e gerar uma simulação condizente com o que de fato ocorreu no período analisado
