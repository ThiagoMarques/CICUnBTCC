
Entender a forma que a pandemia se comporta é fundamental para análises de controle e redução de danos. Foi utilizada a Ferramenta QGIS para o mapeamento e a Linguagem GAMA e COMOKIT para a execução de simulações baseada em agentes. Foram realizados estudos sobre duas regiões no DF durante um determinado espaço de tempo. O primeiro estudo foi a região da Santa Luzia, uma comunidade irregular com imprecisão de dados localizada na Cidade Estrutural em Brasília-DF e o segundo estudo foi no Condomínio RK, o mais regular e melhor estabelecido, localizado em Sobradinho, Brasília-DF. O objetivo dos estudos foi mostrar a disseminação do Covid-19 nesses locais que passaram por políticas restritivas, como exemplo o fechamento de escolas e lockdown a fim de gerar simulações condizentes com o que de fato ocorreu no período analisado.
