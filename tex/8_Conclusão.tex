As principais perguntas feitas para a construção desse trabalho foram:

\begin{enumerate}
\item Como é possível construir uma simulação multiagentes de uma epidemia do COVID-19 em um determinado espaço geográfico, visando subsidiar a construção de políticas públicas no Brasil ? 
\item Quais são as características, possibilidades e dificuldades computacionais para a realização dessas simulações ? 
\item Como é possível a construção de simulações que correspondam de fato ao que aconteceu ou acontecerá ?
\item Que dificuldade precisam ser contornadas para que essas possibilidades sejam alcançadas ?
\end{enumerate}


A construção de uma simulação utilizando a Modelagem Baseada em Agentes(MBA) no Distrito Federal, mais especificamente em 2 setores conhecidos como condomínio RK e Santa Luzia, demonstrou-se viável no auxilio da elaboração de politicas publicas. Uma vez que foi possível demonstrar que a utilização de políticas restritivas ou a sua ausência possui ligação direta com a propagação do vírus e quantidade de casos de COVID-19. 

Foi observado nas execuções das simulações que os cenários que possuíam algum tipo de politica restritiva durante a propagação da doença tendiam a ter curvas mais achatadas e controladas, levando isso para mundo real, tornaria mais fácil controlar a propagação do vírus com as politicas restritivas. Já os modelos que não possuíam políticas restritivas demonstraram uma tendência maior a ter uma curva acentuada que pode ser entendida como uma "explosão" de casos simultâneos, gerando a uma possível sobrecarga do sistema de saúde.    

Para MBAs são necessários diversos agentes, interações, parâmetros e entender tudo isso na sua construção, além do papel de cada um e seu peso na construção do modelo, é o que torna mais difícil na criação de um modelo. Por outro lado são justamente essas características que fazem com que possam surgir modelos cada vez mais refinados sobre situações reais. Além do que cada modelo é essencialmente único pela forma com que as MBAs são construídas e essa flexibilidade na capacidade de alterar os níveis de descrição e agregação viabiliza infinitos modelos \cite{bonabeau2002agent}.

As possibilidades para o GAMA são virtualmente infinitas, já que possui boa integração com CSV, Mapas, Inteligencia Artificial entre muitos outros. Por outro lado a medida que a complexidade das simulações escala, nota-se a necessidade de mais recursos computacionais na sua execução, isso demonstra que existem limitações para computadores comuns no processamento dessas simulações, gerando uma possível limitação para esse tipo de tecnologia em computadores comuns\cite{gamaplataform} .

A construção de um modelo que corresponda a um processo ou evento do mundo real, decorre da análise das características do objeto em analise, com isso é possível fazer a elaboração dos agentes, suas interações e relacionamentos e o ambiente dos agentes. Após essa formulação, a implementação passa a depender de mecanismos computacionais os quais podem ser uma linguagem de programação ou um kit de modelagem\cite{macaltutorial:online}. No caso o kit utilizado como suporte no desenvolvimento desse trabalho foi o COMOKIT que forneceu certos parâmetros genéricos que foram posteriormente ajustados para adequar-se melhor a realidade do Condomínio RK.

As principais dificuldades na elaboração da modelagem iniciam-se ainda na fase de obtenção de dados da região ou fenômeno que serão analisados. Pois sem isso torna muito difícil construir um modelo válido, posteriormente é necessário entender quais agentes serão importantes, suas interações, ambientes e demais parâmetros do modelo, essa sensibilidade na definição de todos esses elementos já tornar os modelos essencialmente únicos. Tendo tudo isso bem estabelecido e programado, ocorre então a analise de diversos resultados que guiarão nas mudanças do peso das variáveis para poder tornar o modelo mais preciso e adequar-se a necessidades reais.

Conforme ocorreu o desenvolvimento de hardware e software nas ultimas décadas, o mundo passou por processos que facilitaram o avanço das tecnologias em diversas áreas. Desde a segunda metade do século XX, as simulações passaram a ser exploradas nas mais diversas áreas tanto do comportamento humano quanto em fenômenos naturais. Sua adaptabilidade permite, através da previsão de tendências comportamentais, antecipar os resultados esperados e possibilitar maneiras de melhorá-los. Sua praticidade, custo, utilidade e bons resultados podem ser usados em contexto educacional para elevar a qualidade do ensino transcendido aos discentes com base nos desempenhos obtidos anteriormente e suas variabilidades, incorporando a situação que trouxer a melhor performance.

As simulações voltadas ao ensino podem ser usadas principalmente para integrar o conhecimento teórico com pratico, em situações que exigem cuidados extras o aluno poderá ser avaliado sem riscos por se tratar de uma simulação. Essa testagem em cenários mais realísticos auxilia na fixação dos conceitos teóricos e torna o aluno apto ao desenvolvimento de raciocínio rápido e correto. O tempo para a aprendizagem pode ser melhor administrado visto que podem ser efetuados ciclos infinitos do mesmo processo, o que torna mais dinâmicos e menos desgastante para o professor e para o aluno


