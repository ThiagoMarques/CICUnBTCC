\section{Utilidade Potencial da Simulação em Políticas de Saúde Pública}

O COVID-19 Modeling Kit (COMOKIT) pode ser utilizado para analisar e comparar intervenções contra a epidemia de COVID-19 em uma cidade ou região. O caso do COVID-19 pelo mundo foi de rápida disseminação do vírus na população mundial, forçando as autoridades de saúde pública a aplicarem políticas restritivas experimentais para conter o vírus.

Os desenvolvedores projetaram o kit de modelagem para ser genérico, escalável e portátil, sendo aceito em vários contextos sociais e geográficos. Criado para para entender, analisar e comparar os impactos das políticas de mitigação contra a epidemia de COVID-19 veio de uma necessidade urgente de ferramentas e metodologias para analisar rapidamente a eficácia das respostas contra a COVID-19 em diferentes comunidades e contextos. Nessa perspectiva, a modelagem computacional aparece como uma alavanca inestimável, pois permite a exploração de uma série de estratégias de intervenção antes da fase potencial de implementação de campo \cite{ArtigoComokit:online}. 

A simulação dos modelos é baseado na transmissão de pessoa para pessoa, no contágio do ambiente e possui um modelo de evolução do estado epidemiológico individual,um modelo de passo de tempo de 1 hora baseado em agenda de mobilidade humana e um modelo de intervenção \cite{ArtigoComokit:online}.

A análise de políticas públicas pode nos ajudar a compreender qual das opções disponíveis levam aos resultados mais satisfatórios. Para poder comparar políticas ou resultados precisamos entender como uma ação gera uma mudança e ser capazes de chegar a um acordo sobre o que é desejado \cite{gentile2015modelos}.

Dada a própria complexidade do menor dos sistemas sociais, essa análise não é trivial. Sistemas sociais compreendem indivíduos autônomos que não se comportam de forma perfeitamente racional e que têm diferentes modelos mentais explicativos de como a sociedade funciona. Sistemas sociais não se comportam de formas determinísticas que se prestem a uma simples análise de planilha ou a uma formulação 
 matemática fechada no nível causal \cite{gentile2015modelos}.

Políticas públicas operam em um ambiente altamente complexo – além do que pode ser controlado ou observado de forma determinística. Na formulação de políticas públicas, devemos considerar mais adiante do que de fato podemos controlar e examinar atentamente o que podemos influenciar. Temos de ser capazes de capturar explicitamente as hipóteses causais subjacentes das propostas de política, de modo que possamos experimentar e fornecer aos stakeholders uma forma de compartilhar e testar suas próprias hipóteses e ideias com outros de 
maneira analiticamente defensável. A modelagem de simulação tem o potencial de fornecer essa possibilidade \cite{gentile2015modelos}.

Portanto, o COMOKIT foi projetado para ser modular e flexível o suficiente para permitir que modeladores e usuários representem diferentes estratégias e estudem seus impactos em múltiplos cenários sociais, epidemiológicos ou econômicos \cite{ArtigoComokit:online}. O seu uso em políticas públicas pode fornecer dados relevantes para construção de hipóteses de mitigação do contágio do COVID-19.


\section{Políticas públicas baseadas em evidências}

Política pública é um conjunto de metas definido por governantes, com diferentes graus de participação dos governados, com o objetivo de solucionar ou prevenir problemas sociais, que se tornam uma agenda política \cite{estrategicasciencia}.

A política baseada em evidência (PBE) é uma prática que tem reunido consenso relativamente à sua necessidade de implementação junto da população e dos gestores responsáveis por tomarem decisões políticas \cite{ramos2018}.

A importância da implementação dessa prática no setor público – i.e. do uso da evidência para formular políticas públicas - é fundamental pelo carácter ético de missão pública que é inerente à atividade política. A utilização de evidência no processo de decisão política poderá, assim assegurar, tanto quanto possível, que as decisões irão ao encontro de necessidades reais. A utilização de evidência no processo político pode melhorar todo o sistema, nomeadamente através do apoio na identificação acurada de necessidades de uma população, da otimização de recursos públicos, da gestão de serviços adequados, minimizando efeitos que podem surgir de decisões pouco fundamentadas em ciência \cite{santos2020ciencia}.

Dentre as possíveis abordagens às políticas públicas, é possível assumir uma perspectiva prescritiva (como deve ser feita a política pública) ou descritiva (como é feita a política pública). A abordagem que enfatiza o uso das evidências (evidence-based policy) é uma abordagem prescritiva, e que pode ser compatibilizada com a compreensão descritiva das políticas em termos de ciclos, que é uma das mais tradicionais do campo de políticas públicas. O modelo dos ciclos de políticas públicas possui cinco passos ou cinco etapas: montagem de agenda, formulação da política, tomada de decisão, implementação da política e avaliação da política \cite{de2018politicas}.

Na primeira etapa, a montagem de agenda, o problema está sendo reconhecido pelos atores políticos e há diversidade de escolhas e soluções. Na segunda etapa, formulação da política, excluem-se as opções não executáveis e cada um dos atores políticos envolvidos age tentando fazer com que a sua solução favorita do problema seja a escolhida. Na terceira etapa, relativa à tomada de decisão, o governo ou o responsável, efetivamente, escolhe uma das soluções, que será concretizada na quarta etapa, relativa à implementação da política. Nesse momento, a decisão é colocada em ação, usando ferramentas disponíveis à administração pública, que visem alterar distribuição de recursos e serviços na sociedade. A quinta etapa, por fim, é a de avaliação da política, quando tanto atores estatais quanto sociais monitoram os resultados alcançados pela política \cite{de2018politicas}.

Esse modelo baseado em etapas ajuda a vislumbrar a política em seu aspecto prático, possibilitando, inclusive, que haja critérios específicos para cada uma das fases: aquilo que é necessário saber e fazer na fase de implementação certamente diverge daquilo que se precisa na fase de avaliação, até mesmo em termos de capacidades e agentes. No entanto, como qualquer modelo, essa categorização das políticas em ciclos é uma idealização do processo real no campo de atuação. Na prática, essas fases se “apresentam misturadas” e as
“sequências se alternam” \cite{de2018politicas}.

As evidências científicas em geral podem ser utilizadas em diversas etapas do ciclo de políticas públicas, e de diversas formas. Uma das formas possíveis de conexão entre evidência e política pública consiste em usar as próprias políticas como fonte de evidências. Assim, os atores políticos aprenderiam com o passado e utilizariam políticas anteriores como base para novos desenhos de políticas. As evidências devem vir, portanto, antes da proposição das políticas, por meio de análises sistemáticas dos efeitos de políticas similares.
Não se pode recusar a utilidade desse tipo de evidência, e claro é que a pesquisa em política tem muito a ganhar seguindo uma sequência em que as intervenções sociais são instadas a tentar, tentar e tentar de novo resolver os problemas com os quais a sociedade moderna se confronta. Essa é a raison d’être por trás da atual explosão do interesse nas políticas baseadas em evidências \cite{de2018politicas}.

A Pandemia da Covid-19 afetou sociedades de todo o mundo, fazendo com que líderes adotassem diversas medidas de urgência, quer seja para prevenir e reduzir a velocidade de contágio, quer seja para garantir a sobrevivência e a dignidade de seu povo diretamente afetado pelas restrições necessárias. Nesse processo, variadas políticas públicas e instrumentos foram utilizados para superar a crise causada. Guiados por órgãos internacionais, como a OMS (Organização Mundial da Saúde) e seus grupos de trabalhos específicos para a Covid-19, algumas das ações mais imediatas foram utilizadas em sociedades e países diferentes. Sistemas de ensino à distância, home-office, fechamento de fronteiras, alteração de horários de comércio, uso obrigatório de máscaras, distanciamento social são exemplos de recomendações que seguiram de forma padronizada em diversos países e sociedades durante a pandemia. Outras alternativas adotadas, no entanto, mais particulares de cada país e sistemas políticos, podem ser foco de estudos futuros que busquem entender como a pandemia e as mudanças no sistema de reconhecimento de problemas e seleção de alternativas conseguiram inserir difíceis pautas na onda da crise \cite{brasil2020estudos}.

As raízes do policy learning podem ser observadas nos estudos de Heclo (1974) e na profunda difusão e mapeamento dessa lente analítica feitos por Hall (1993) e Dunlop e Radaelli (2013). Nessa abordagem, o destaque reside na aquisição e utilização do conhecimento e de evidências no envolvimento e no aprendizado baseado em valores, crenças e ideias para a análise do processo de produção das políticas públicas. De acordo com Hall (1993, p. 278), o aprendizado “ocorre quando os indivíduos assimilam uma informação nova, incluindo aquela
baseada em experiência, e a aplicam em suas ações subsequentes. [...] O aprendizado é observado quando a política muda como resultado desse processo” \cite{brasil2020estudos}.

\section{Utilidade da Simulação na Disseminação de Informação}

A informação sobre covid-19 se espalhou rapidamente entre diversos meio de comunicação, incrementado por ser uma sociedade altamente conectada no ano da pandemia e com a atenção da população direcionada totalmente voltado ao tema da pandemia de COVID-19, que viveu restrições e dependiam de informações oficiais para abertura de comércios, retorno às aulas, convívio entre as pessoas, com restrições de visitas, compras e uma dependência de vacina para até mesmo sua sobrevivência pessoal.

Mas infelizmente, sofremos também de um surto de desinformação (ou seja, informações enganosas, deliberadamente circuladas para causar danos) sobre a COVID-19 se espalhou de maneira rápida, ampla e a baixos custos pela Internet, colocando vidas em risco e dificultando a recuperação \cite{Ocombate55:online}.

Os efeitos nocivos da desinformação sobre a COVID-19 não podem ser subestimados. Dados \cite{Navigati3:online} da Argentina, Alemanha, Coreia, Espanha, Reino Unido e Estados Unidos mostram que cerca de uma em cada três pessoas dizem ter visto informações falsas ou enganosas relacionadas à COVID-19 nas mídias sociais. Pesquisas também mostraram que a desinformação a respeito da COVID-19 é disseminada de forma significativamente mais ampla do que as informações sobre o vírus por fontes autorizadas como a Organização Mundial de Saúde (OMS) e os Centros para Controle e Prevenção de Doenças dos Estados Unidos. Ao questionar fontes e dados oficiais e convencer pessoas a tentarem tratamentos falsos, a disseminação de má informação e de desinformação levou pessoas a ingerir curas caseiras fatais, ignorar regras de distanciamento social e de lockdown, e a não usar máscaras protetoras, prejudicando a eficácia de estratégias de contenção \cite{Ocombate55:online}.

Porém a disseminação de informação correta e transferência em políticas públicas, a forma pela qual entidades internacionais, como a OMS, coordenaram e difundiram recomendações ao longo da pandemia da Covid-19, servindo de base não apenas para a difusão de informação, de dados oficiais e acompanhamento do número de casos e de mortes no mundo, mas também analisando e recomendando ações baseadas em experiências anteriores. O aprendizado sobre a doença bem como o aprendizado sobre as políticas empregadas pelos países que vivenciaram a primeira onda de contágios e mortes, como China, Espanha e Itália, produziram conhecimento e evidências sobre as ações tomadas, apontando e separando o que funcionou e o que não surtiu efeito no controle do contágio e de suas consequências. Em muitos países, como Finlândia e Nova Zelândia, por exemplo, a antecipação na produção de políticas restritivas, de isolamento, fechamento de fronteiras e informações sobre uso de máscaras, higienização de mãos, por exemplo, fez com que o impacto da pandemia, nesses países, fosse drasticamente inferior quando comparado à situação daqueles que vivenciaram a doença na primeira onda, ou daqueles países cujos governantes optaram por não acatar as recomendações provenientes do aprendizado e das evidências anteriores, como os casos do Brasil e dos Estados Unidos \cite{brasil2020estudos}. 

\section{Utilidade da Simulação na Educação, Ensino, Aprendizagem e Treinamento}

%\todo{Tendo em vista o que foi abordado em 1.2.2, escrever aqui as reflexões e impactos dos trabalhos de vocês para o tópico, possivelmente no tema do estudo da epidemiologia. Citar os guardiões da saúde, que foram usados na UnB durante a pandemia. Citar as possibilidades de uso no curso de licenciatura da UnB

Em 2020, devido à disseminação do Covid-19, o bloqueio de escolas e universidades mudou os métodos de ensino em 61 países ao redor do mundo. Foi uma mudança do encontro presencial para educação online \cite{UNESCO2020:online}.De repente, milhões de membros do corpo docente em um lado, e os alunos correspondentes, por outro, têm usado a internet como a única ferramenta possível para interação docente \cite{BAO2020:online}.

No Brasil, a paralisação das atividades nas escolas e nas universidades não significou, necessariamente, um período de folga para professores e alunos. Em algumas redes públicas, a suspensão das atividades presenciais efetivamente traduziu-se na suspensão das atividades de ensino, ainda que em muitas esteja havendo atividades remotas. Contudo, escolas e universidades privadas, inclusive na Educação Infantil, determinaram que as atividades presenciais deveriam ser transpostas, por meio de ferramentas digitais, para um modelo de educação remota enquanto durasse a crise sanitária. Tal decisão recebeu, inclusive, suporte legal do Ministério da Educação (MEC) \cite{saraiva2020educacao}.

Contudo, presenciamos uma realidade em que a sociedade, de modo geral, não estava preparada para assumir  a  tecnologia  de  forma  tão  contundente  em  sua  vida cotidiana.  Apesar  de  consumir  tecnologia  (bens  de  consumo  como  celular e outros  dispositivos  que  são  acessíveis  ao  trabalhador  comum),  as  pessoas não dependiam exclusivamente da tecnologia, principalmente das plataformas virtuais, para ter acesso à educação, por exemplo \cite{miranda2021formacao}.

O cenário pandêmico nos trouxe um termo encontrado com frequência para designar o tipo de comunicação das ferramentas online: comunicação síncrona e assíncrona. De acordo com Morais e Cabrita (2007), a comunicação síncrona caracteriza-se, como a própria designação sugere, pelo sincronismo da comunicação, ou seja, a troca de informação realiza-se em simultâneo, exigindo que os participantes se encontrem online ao mesmo tempo para poderem comunicar entre si. Na comunicação assíncrona a transmissão de informação ocorre de modo diferido, não exigindo, por isso, a disponibilidade simultânea dos diferentes participantes \cite{morais2007ambientes}.

O modo como o professor e aluno interagem influência diretamente na qualidade do ensino. Mesmo com a disponibilidade de inúmeras ferramentas digitais para auxílio na comunicação síncrona e assíncrona e adaptação de metodologias de ensino, é necessário minimizar impactos no ensino remoto quando comparado ao ensino presencial. A função central do professor não é utilizar recursos digitais de forma aleatória, ou aplicar tecnologias prontas – mas assumir plenamente o papel de construtor e direcionador do conhecimento. Isso implica em organizar e ajustar suas aulas e disciplinas para este novo formato online, de acordo com as necessidades de aprendizagem dos estudantes, buscando alternativas didáticas adequadas e personalizadas que estimulem a participação, a inclusão e a assimilação do conhecimento \cite{santos2021covid}.

Para o estudo de epidemiologia do surto de doença por Covid-19, a simulação computacional pode ajudar a compreender a dinâmica da transmissão do vírus de pessoa para pessoa, visualizar computacionalmente em uma população a fase de incubação do vírus e testar os impactos da quarentena, verificar a manifestação do vírus em indivíduos que usam a máscara como proteção em relação a uma população que não utiliza essa proteção.

Além da compreensão da dinâmica da transmissão do vírus, uma das vantagens de se usar a simulação é a possibilidade de verificar impactos na políticas de quarentena. O período de incubação de SAR-COV-2 é de 14 dias, com mediana de 4 a 6 dias, embora há relatos de períodos de incubação de até 24 dias \cite{kang2020impact}. Períodos de incubação longos podem impactar negativamente sobre os resultados esperados para as políticas de quarentena que objetivam conter a disseminação do vírus \cite{netto2020epidemiologia}. No COMOKIT podemos testar diferentes períodos e colher informações que ajudam a minimizar os impactos da disseminação do vírus e ajustar políticas de quarentena.

A simulação do COMOKIT pode apoiar projetos que necessitam de informações do COVID-19 em uma população. O uso na vigilância comunitária, que se caracteriza pelo monitoramento com participação da comunidade \cite{oliveira2021equipe}, pode gerar dados que auxiliam a compreensão das condições de saúde de uma população de maneira sistemática e contínua.

Um exemplo é a aplicação Guardiões da Saúde, um aplicativo gratuito para dispositivos móveis, tanto para IOS quanto Android. Esse aplicativo foi lançado em um curto prazo para implementar um mecanismo de vigilância comunitária em saúde, diante do acelerado ritmo da pandemia, como parte do controle da epidemia e para lidar com o surgimento de uma infodemia com um forte componente de notícias falsas. \cite{oliveira2021equipe}.

O uso do aplicativo possibilitou a interação entre integrantes de cursos diversos, ajudou a elaborar estratégias para enfrentar a infodemia e notícias falsas e a responder à necessidade de orientação da comunidade externa e acadêmica sobre o novo Coronavírus. Composto por uma equipe multiprofissional de Tecnologia da Informação que visa atender as demandas dos usuários, Análise Epidemiológica responsável pela produção das postagens e Comunicação que elabora a estrutura textual das postagens \cite{oliveira2021equipe}. O COMOKIT pode apoiar as três equipes.

Para Tecnologia da Informação, a simulação pode trazer os benefícios de exportação de dados e implementação em aplicativos. De acordo com a documentação do GAMA Platform, software usado na construção do COMOKIT, o tipo de arquivo ao salvar pode ser "shp", "json" and "kml" para dados espaciais vetoriais (agentes e geometrias), "asc" e "geotiff" para dados espaciais raster (grid), "image" para imagem, "dimacs", "dot", "gexf", "graphml", "gml" and "graph6" para gráficos, "text" and "csv" \cite{GamaWiki:online}. Isso permite integrar gráficos, imagens e dados da simulação no aplicativo.

O uso da simulação na Análise Epidemiológica e na Comunicação permite compreender a dinâmica da transmissão do vírus e produzir dados com diferentes parâmetros em uma população. Divulgar orientações para a comunidade externa e acadêmica sobre o novo Coronavírus utilizando uma ferramenta com bases cientificas reforça a credibilidade de informações, podendo assim minimizar orientações falsas que foram disseminadas sem embasamento científico.

A simulação do COMOKIT na licenciatura da Computação além de uma ferramenta poderosa capaz de simular políticas públicas para Covid-19, pode ser usada como uma ferramenta pedagógica no ensino de epidemiologia e indo além do ensino da simulação, aproxima os alunos de um software capaz de apoiar decisores a responderem as melhores políticas de controle da disseminação do COVID-19.











