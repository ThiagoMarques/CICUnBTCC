De que forma a simulação multiagente pode ser útil para a construção de políticas públicas para tratamento de epidemias no DF?

\section{Dificuldades Computacionais}

O COMOKIT pode ser utilizado em computadores pessoais, o que abre um leque de possibilidades para o seu uso com poucas limitações. Porém, a medida que simulações se tornam complexas, aumenta o consumo de recursos computacionais do equipamento. De acordo com a documentação técnica do GAMA (GIS Agent-based Modeling Architecture), é possível importar um grande número de tipos diferentes de dados como arquivos de textos, CSV, shapefiles, OSM (open street map data), imagens, SVG, arquivos 3D entre outros. Isso nos mostra que a complexidade de simulações pode aumentar dependendo da quantidade de dados utilizados para realizar uma simulação \cite{gamaplataform} .

Portanto, se houver uma necessidade de simular um grande número de dados complexos implicará diretamente em um sobrecarregamento dos recursos computacionais da máquina que deverá rodar a simulação.


\section{Dificuldades com a  simulação}

As simulações fornecidas pelo COVID-19 Modeling Kit (COMOKIT) não leva em consideração todos os aspectos reais enfrentados no período de 2020 a 2021 durante a pandemia de COVID-19. Apesar de toda a flexibilidade dos modelos, não foi possível simular todos os aspectos reais da simulação devido a complexidade não somente do cenário enfrentado de uma doença nova para a população, mas também do grande número de possibilidades e parâmetros que o COMOKIT fornece.

\section{Dificuldades com dados Geográficos}

A dificuldade para mapear o bairro Santa Luzia localizado na Estrutural em Brasília-DF, uma região periférica, sem auxílio do poder público se agravou no período de disseminação do COVID-19, onde não havia fontes para compartilhar informações. O bairro Santa Luzia foi alvo inicial da pesquisa não havia disponível dados demográficos e dados populacionais para desenvolvimento da pesquisa.


\section{Dificuldades com dados epidemiológicos}

Observou-se uma facilidade em obter dados referentes a uma das abordagens do estudo que foi o condomínio RK, isso ocorreu por já existir uma equipe diretamente relacionada ao local que estava coletando dados epidemiológicos durante um espaço de tempo nessa região. Por outro lado a inexistência do acompanhamento da comunidade da Santa Luzia por uma equipe somada com a dificuldade do acesso, imprecisão de dados geográficos e epidemiológicos tornou o setor da Santa Luzia difícil de avaliar e montar um modelo.





