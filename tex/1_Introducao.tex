Os vírus da família do coronavírus eram considerados patógenos de pouca importância. Em 2002, isso começou a mudar com o surgimento de um vírus com alta taxa de mortalidade que surgiu na China e causava uma síndrome respiratória aguda e é de onde vem seu nome \textbf{Sars-CoV}. Em poucos meses esse vírus foi controlado e erradicado, apesar de ter deixado mais de 8000 pessoas infectadas e 800 mortos.

Em dezembro de 2019 foi identificado o primeiro caso de um vírus novo em Wuhan, na China, esse novo vírus é de difícil controle e tem alta disseminação, inclusive é transmitido também por indivíduos assintomáticos, fatores que influenciaram na sua propagação global. Por ter similaridades com o coronavírus que ocorreu em 2002, recebeu um nome parecido \textbf{Sars-CoV-2}, mais conhecido como Covid-19. 

%\url{http://www.escritacientifica.com/}

%%%%%%%%%%%%%%%%%%%%%%%%%%%%%%%%%%%%%%%%%%%%%%%%%%%%%%%%%%%%%%%%%%%%%%%%%%%%%%%%
%%%%%%%%%%%%%%%%%%%%%%%%%%%%%%%%%%%%%%%%%%%%%%%%%%%%%%%%%%%%%%%%%%%%%%%%%%%%%%%%
%%%%%%%%%%%%%%%%%%%%%%%%%%%%%%%%%%%%%%%%%%%%%%%%%%%%%%%%%%%%%%%%%%%%%%%%%%%%%%%%
\section{Questão de pesquisa}%
As perguntas fundamentais que foram base para essa pesquisa.

É possível construir uma simulação baseada em agentes da pandemia do covid em um determinado espaço geográfico ?  Essa simulação corresponde de fato ao que aconteceu ? Alguma dificuldade que não pode ser contornada ?


%%%%%%%%%%%%%%%%%%%%%%%%%%%%%%%%%%%%%%%%%%%%%%%%%%%%%%%%%%%%%%%%%%%%%%%%%%%%%%%%
%%%%%%%%%%%%%%%%%%%%%%%%%%%%%%%%%%%%%%%%%%%%%%%%%%%%%%%%%%%%%%%%%%%%%%%%%%%%%%%%
%%%%%%%%%%%%%%%%%%%%%%%%%%%%%%%%%%%%%%%%%%%%%%%%%%%%%%%%%%%%%%%%%%%%%%%%%%%%%%%%
\section{Relevância e Justificativa}%

Simulações baseadas em agentes pode ser entendida como simulação de agentes individuais, os quais geram algo mais complexo a partir da ação de cada indivíduo, dessa forma um fenômeno em grande escala pode ser mais facilmente entendido se observado cada agente de forma individual.

Um bom exemplo seria a movimentação de um cardume de peixes, trânsito em uma cidade, disseminação de uma doença e voo de pássaros. Inclusive esse último possui um modelo muito bem elaborado na ferramenta Netlogo que utiliza-se de 3 regras “alinhamento”, “coesão” e “separação” na movimentação individual de cada pássaro para obter o movimento coordenado de um bando de pássaros.

Neste trabalho será utilizado a ferramenta GAMA devido a sua praticidade tanto com dados georeferenciados quanto por ser uma linguagem de alto nível, a qual possibilita a criação intuitiva de agentes. Além disso possui diversas interfaces, tanto 2D quanto 3D, para a visualização de resultados com painel controláveis pelo usuário.

A história mostra que a humanidade já passou por outras pandemias e por conta da atual globalização existe uma enorme possibilidade de acontecer outra como a atual da Covid-19. Por isso é importante a construção de modelos para analisar parâmetros e padrões das pandemias com objetivo de ajudar tanto no controle da atual quanto na redução de danos de futuras pandemias. 

Este trabalho baseado na linguagem GAMA busca obter um modelo ao qual forneça dados aproximados ao que de fato aconteceu em uma região de Brasília, com isso é possível buscar meios e novas perspectivas para o enfrentamento da Covid-19.

**Também será mostrado nesse trabalho a tentativa de elaboração do mesmo modelo, porém em uma região diferente de Brasília com pouco acesso a informação de dados geográficos, qualitativos e quantitativos.


%%%%%%%%%%%%%%%%%%%%%%%%%%%%%%%%%%%%%%%%%%%%%%%%%%%%%%%%%%%%%%%%%%%%%%%%%%%%%%%%
%%%%%%%%%%%%%%%%%%%%%%%%%%%%%%%%%%%%%%%%%%%%%%%%%%%%%%%%%%%%%%%%%%%%%%%%%%%%%%%%
%%%%%%%%%%%%%%%%%%%%%%%%%%%%%%%%%%%%%%%%%%%%%%%%%%%%%%%%%%%%%%%%%%%%%%%%%%%%%%%%
\section{Metodologia}%

Nessa pesquisa foram analisadas duas regiões distintas do DF, para tal foi utilizado as ferramentas QGIS para mapeamento da região e GAMA para a modelagem. 

Para criar um modelo no GAMA que utilize-se de uma área geográfica é necessário possuir ou criar os arquivos shapefile, no caso de Santa Luzia por ser um assentamento irregular, poucos dados estavam disponíveis.

Tendo isso em vista, a região de Santa Luzia foi mapeada manualmente no OpenStreetMap e posteriormente foi feita a exportação do arquivo para o QGIS, com isso foi possível gerar o shapefile para exportar para o GAMA. 

Na construção do modelo baseado em agentes é necessário definir quais serão os agentes, seus parâmetros, comportamentos e interações com outros agentes. Observando-se isso foi realizado um estudo sobre a região para identificar os principais agentes e suas relações na comunidade, locais de lazer, escolas, hospitais, comércio, são agentes que possuem um impacto significativo na simulação.

Para auxiliar na elaboração da modelagem da Covid-19 na Santa Luzia foi utilizado o COMOKIT (Kit de modelagem da Covid-19) feito no GAMA para comparar diversos tipos de intervenções possíveis, durante as adaptações do modelo foram encontradas diversas dificuldade em obter dados de infectados e curados pela Covid na região da Santa Luzia o que tornou inviável a continuação da modelagem.

Com esse obstáculo foi necessário procurar uma nova região a qual pudesse fornecer de forma confiável esses dados que posteriormente ficou definido como Condomínio RK.

No Condomínio RK já foi possível obter dados como shapefile e diversos dados relacionados a Covid na região, dessa forma foi possível fazer ajustes e adaptações no COMOKIT e verificar que o modelo obteve um comportamento similar ao ocorrido de fato na região.

%%%%%%%%%%%%%%%%%%%%%%%%%%%%%%%%%%%%%%%%%%%%%%%%%%%%%%%%%%%%%%%%%%%%%%%%%%%%%%%%
%%%%%%%%%%%%%%%%%%%%%%%%%%%%%%%%%%%%%%%%%%%%%%%%%%%%%%%%%%%%%%%%%%%%%%%%%%%%%%%%
%%%%%%%%%%%%%%%%%%%%%%%%%%%%%%%%%%%%%%%%%%%%%%%%%%%%%%%%%%%%%%%%%%%%%%%%%%%%%%%%
\section{Conteúdo da Monografia}%
[...]

%\section{Normas CIC}
% \href{http://monografias.cic.unb.br/dspace/normasGerais.pdf}{Política de Publicação de Monografias e Dissertações no Repositório Digital do CIC}%
% \href{http://monografias.cic.unb.br/dspace/}{Repositório do Departamento de Ciência da Computação da UnB}

% \href{http://bdm.bce.unb.br/}{Biblioteca Digital de Monografias de Graduação e Especialização}