De que forma a simulação multiagente pode ser útil para a construção de políticas públicas para tratamento de epidemias?


Este trabalho relata os casos de desenvolvimento de duas simulações baseadas em agentes, para simulação casos de COVID-19 na população de dois territórios do DF.

%%%%%%%%%%%%%%%%%%%%%%%%%%%%%%%%%%%%%%%%%%%%%%%%%%%%%%%%%%%%%%%%%%%%%%%%%%%%%%%%
%%%%%%%%%%%%%%%%%%%%%%%%%%%%%%%%%%%%%%%%%%%%%%%%%%%%%%%%%%%%%%%%%%%%%%%%%%%%%%%%
%%%%%%%%%%%%%%%%%%%%%%%%%%%%%%%%%%%%%%%%%%%%%%%%%%%%%%%%%%%%%%%%%%%%%%%%%%%%%%%%
\section{Questão de pesquisa}%
As perguntas fundamentais que foram base para essa pesquisa.

Como é possível construir uma simulação multiagentes de uma epidemia do COVID-19 em um determinado espaço geográfico, visando subsidiar a construção de políticas públicas no Brasil ? Quais são as características, possibilidades e dificuldades computacionais para a realização dessas simulações? Como é possível a construção de simulações que correspondam de fato ao que aconteceu ou acontecerá ? Que dificuldade precisam ser contornadas para que essas possibilidades sejam alcançadas ?


%%%%%%%%%%%%%%%%%%%%%%%%%%%%%%%%%%%%%%%%%%%%%%%%%%%%%%%%%%%%%%%%%%%%%%%%%%%%%%%%
%%%%%%%%%%%%%%%%%%%%%%%%%%%%%%%%%%%%%%%%%%%%%%%%%%%%%%%%%%%%%%%%%%%%%%%%%%%%%%%%
%%%%%%%%%%%%%%%%%%%%%%%%%%%%%%%%%%%%%%%%%%%%%%%%%%%%%%%%%%%%%%%%%%%%%%%%%%%%%%%%
\section{Relevância e Justificativa}%

%Finalidade: nosso trabalho é uma pesquisa básica e estratégica, ou seja, pode ser usado como base para resolver outros problemas%

%Objetivo da nossa pesquisa é descritiva e exploratória pois levantamos dados que não estão em livros%

%Abordagem quali-quantitativa pois usamos métodos estatísticos e simulações%

%Método hipotético-dedutivo pois usamos a simulação para verificar o comportamento da população sob restrições da pandemia de COVID-19%

\subsection{Compreensão de Fenômenos Complexos}

Simulações baseadas em agentes pode ser entendida como simulação de agentes individuais, os quais geram algo mais complexo a partir da ação de cada indivíduo, dessa forma um fenômeno em grande escala pode ser mais facilmente entendido se observado cada agente de forma individual.

Um bom exemplo seria a movimentação de um cardume de peixes, trânsito em uma cidade, disseminação de uma doença e voo de pássaros. Inclusive esse último possui um modelo muito bem elaborado na ferramenta Netlogo, uma linguagem de programação e um ambiente de desenvolvimento integrado baseada em agentes, que utiliza-se de 3 regras “alinhamento”, “coesão” e “separação” na movimentação individual de cada pássaro para obter o movimento coordenado de um bando de pássaros.

\subsection{Ensino, Aprendizagem e Treinamento}

São vários os exemplos de uso de simulação multiagente para potencializar momentos de ensino-aprendizagem:
\begin{description}
\item [Comportamento Social e COVID-19] Em \cite{santoro_building_2020}, os autores empregaram simulações da difusão da COVID-19 como suporte ao planejamento de comportamentos futuros de estudantes e professores em uma cidade, visando a continuidade de aulas por meio de encontros online; 
\item [Aprendizagem de Inteligência Artificial] Em \cite{juarez_analysis_2022} os autores usaram simulação da Pandemia da COVID-19 para potencializar o aprendizado da inteligência artificial; Em \cite{filho_proposal_2007} os autores usaram simulação multiagente para motivar atrair e motivar pesquisadores no campo do planejamento e aprendizagem em inteligência artificial;
\item [Comportamento em Sala de Aula] Em \cite{zhao_research_2011} os autores usaram simulação multiagente para compreender problemas comportamentais de estudantes em sala de aula;
\item [Educação Especial] Em \cite{galitsky_computational_2013} o autor empregou uma ferramenta de simulação de estados mentais junto a crianças autistas, para fins de reabilitação;
\item [Aprendizagem de Conceitos Científicos]
Em \cite{wilensky_thinking_2006} os autores apresentam as possibilidades do uso de simulação multiagente por estudantes do ensino médio, para compreensão da inter relação entre conceitos de biologia em múltiplos níveis de fenômenos, desde o aspecto imunológico até o ecológico; Em \cite{kottonau_interactive_2011} o autor descreveu como o uso de simulação multiagente contribuiu para a aprendizagem do conceito de osmose, entre estudantes universitários de biologia; Em \cite{bollen_simsketch_2013} os autores propõem o uso de uma ferramenta SimSketch, para propiciar a construção de simulações e a aprendizagem de fenômenos científicos;
\item [Tomada de Decisão em Crises] Em \cite{tena-chollet_design_2016} os autores empregaram um ambiente de simulação multiagente para apoiar a melhoria de habilidades de tomada de decisão de pessoas que vivenciam  situações de crise, onde o stress, a incerteza sobre o desdobramento de fatos, as dificuldades de comunicação e a necessidade de antecipação de fatos afetam a qualidade da decisão; 
\item [Tomada de Decisão em Negócios] Em \cite{hishiyama_business_2015} os autores usam simulação multiagente de uma empresa fabril para ensinar técnicas de tomada de decisão a gestores em formação; 
\item [Tomada de Decisão em Políticas Públicas de Segurança] Em \cite{furtado_multiagent_2006} os autores usando um sistema tutorial de simulação multiagente para orientar melhoria de tomada de decisão em contexto de segurança pública;
\end{description}


\subsection{Experiência Prática e Aprendizagem Ativa}

Neste trabalho será utilizado a ferramenta COMOKIT/GAMA devido à sua praticidade tanto no uso de dados georreferenciados quanto por dispor uma linguagem de alto nível, a qual possibilita a criação intuitiva de agentes. Além disso possui diversas interfaces, tanto 2D quanto 3D, para a visualização de resultados com painel controláveis pelo usuário.

A história mostra que a humanidade já passou por outras pandemias e por conta da atual globalização existe uma enorme possibilidade de acontecer outra como a atual da COVID-19. Por isso é importante a construção de modelos para analisar parâmetros e padrões das pandemias com objetivo de ajudar tanto no controle da atual quanto na redução de danos de futuras pandemias. 

Este trabalho baseado na linguagem GAMA busca obter um modelo ao qual forneça dados aproximados ao que de fato aconteceu em uma região de Brasília, com isso é possível buscar meios e novas perspectivas para o enfrentamento da COVID-19.
Também será mostrado nesse trabalho a tentativa de elaboração do mesmo modelo, porém em uma região diferente de Brasília com pouco acesso a informação de dados geográficos, qualitativos e quantitativos.


%%%%%%%%%%%%%%%%%%%%%%%%%%%%%%%%%%%%%%%%%%%%%%%%%%%%%%%%%%%%%%%%%%%%%%%%%%%%%%%%
%%%%%%%%%%%%%%%%%%%%%%%%%%%%%%%%%%%%%%%%%%%%%%%%%%%%%%%%%%%%%%%%%%%%%%%%%%%%%%%%
%%%%%%%%%%%%%%%%%%%%%%%%%%%%%%%%%%%%%%%%%%%%%%%%%%%%%%%%%%%%%%%%%%%%%%%%%%%%%%%%

\section{Objetivos Gerais e Específicos}%

Como objetivo geral nesse trabalho é desenvolver casos de modelagem e simulação espacialmente explícita sobre o impacto da COVID-19, em regiões do Distrito Federal, utilizando-se de uma ferramenta COMOKIT baseada numa linguagem de simulação  em agentes.

Como objetivos específicos: 
\begin{enumerate}
\item Criar simulações (casos);
\item Realizar as simulações e coletar dados;
\item Comparar os dados obtidos com dados reais disponíveis;
\item Identificar questões que afetam a utilidade da simulação ma formulação de políticas de saúde pública.


%\item ferramentas para obter dados de um sistema de informação geográfica livre e aberta;
%\item Identificar ferramentas para as modelagens e simulações;
%\item Identificar regiões as quais são possíveis obter dados e desenvolver um modelo;
%\item Construir o modelo baseado na região selecionada;
%\item Comparar o resultado obtido no modelo com os dados apresentados na região.
\end{enumerate}
%%%%%%%%%%%%%%%%%%%%%%%%%%%%%%%%%%%%%%%%%%%%%%%%%%%%%%%%%%%%%%%%%%%%%%%%%%%%%%%%
%%%%%%%%%%%%%%%%%%%%%%%%%%%%%%%%%%%%%%%%%%%%%%%%%%%%%%%%%%%%%%%%%%%%%%%%%%%%%%%%
%%%%%%%%%%%%%%%%%%%%%%%%%%%%%%%%%%%%%%%%%%%%%%%%%%%%%%%%%%%%%%%%%%%%%%%%%%%%%%%%
\section{Metodologia}%

Nessa pesquisa foram analisadas duas regiões distintas do DF, para tal foi utilizado as ferramentas QGIS para mapeamento da região e GAMA para a modelagem. 

Para criar um modelo no GAMA que faz uso de uma área geográfica é necessário inserir arquivos do tipo \textit{shapefile}, no caso de Santa Luzia por ser um assentamento irregular, poucos dados estavam disponíveis e foi necessário mapear a área utilizando a ferramenta \textit{OpeenStreetMap} e \textit{QGIS}, sendo possível gerar o shapefile para exportar para o GAMA.

Na construção do modelo baseado em agentes é necessário definir quais serão os agentes, seus parâmetros, comportamentos e interações com outros agentes. Observando-se isso foi realizado um estudo sobre a região para identificar os principais agentes e suas relações na comunidade, locais de lazer, escolas, hospitais, comércio, são agentes que possuem um impacto significativo na simulação.

Para auxiliar na elaboração da modelagem da COVID-19 na Santa Luzia foi utilizado o COMOKIT (Kit de modelagem da COVID-19) feito no GAMA para comparar diversos tipos de intervenções possíveis, durante as adaptações do modelo foram encontradas diversas dificuldade em obter dados de infectados e curados pela Covid na região da Santa Luzia o que tornou inviável a continuação da modelagem.

Com esse obstáculo foi necessário procurar uma nova região a qual pudesse fornecer de forma confiável esses dados que posteriormente ficou definido como Condomínio RK.

No Condomínio RK foi possível obter dados como shapefile e diversos dados relacionados a COVID na região, dessa forma foi possível fazer ajustes e adaptações no COMOKIT e verificar que o modelo obteve um comportamento similar ao ocorrido de fato na região.

%%%%%%%%%%%%%%%%%%%%%%%%%%%%%%%%%%%%%%%%%%%%%%%%%%%%%%%%%%%%%%%%%%%%%%%%%%%%%%%%
%%%%%%%%%%%%%%%%%%%%%%%%%%%%%%%%%%%%%%%%%%%%%%%%%%%%%%%%%%%%%%%%%%%%%%%%%%%%%%%%
%%%%%%%%%%%%%%%%%%%%%%%%%%%%%%%%%%%%%%%%%%%%%%%%%%%%%%%%%%%%%%%%%%%%%%%%%%%%%%%%




%\section{Normas CIC}
% \href{http://monografias.cic.unb.br/dspace/normasGerais.pdf}{Política de Publicação de Monografias e Dissertações no Repositório Digital do CIC}%
% \href{http://monografias.cic.unb.br/dspace/}{Repositório do Departamento de Ciência da Computação da UnB}

% \href{http://bdm.bce.unb.br/}{Biblioteca Digital de Monografias de Graduação e Especialização}