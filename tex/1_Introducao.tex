A Organização Pan-Americana da Saúde (OPAS) publicou que em 31 de dezembro de 2019, a Organização Mundial da Saúde (OMS) foi alertada sobre casos de pneumonia na cidade de Wuhan, província de Hubei, na República Popular da China. A nova cepa, até o momento ainda desconhecida a seres humanos de coronavírus foi confirmada em janeiro de 2020.

Ao todo, sete coronavírus humanos (HCoVs) já foram identificados: HCoV-229E, HCoV-OC43, HCoV-NL63, HCoV-HKU1, SARS-COV (que causa síndrome respiratória aguda grave), MERS-COV (que causa síndrome respiratória do Oriente Médio) e o, mais recente, novo coronavírus (que no início foi temporariamente nomeado 2019-nCoV e, em 11 de fevereiro de 2020, recebeu o nome de SARS-CoV-2). Esse novo coronavírus é responsável por causar a doença COVID-19.

No Brasil, foi inicialmente declarado uma Emergência de Saúde Pública de Importância Nacional (ESPIN) de acordo com a Portaria Nº 188, de 3 de fevereiro de 2020 publicada no Diário Oficial da União, a portaria recomendou divulgar informações a população e até o momento não aplicou determinações restritivas para controle da COVID-19, chamada até a data da divulgação da portaria de  2019-nCoV.

De acordo com informações divulgadas pela Organização Pan-Americana da Saúde (OPAS), foi declarado em 11 de março de 2020 com objetivo de interromper a propagação do vírus uma Emergência de Saúde Pública de Importância Internacional (ESPII) – o mais alto nível de alerta da Organização.  

Este trabalho realiza uma simulação baseada em agentes de uma população brasileira de até 10.000 indivíduos que foram submetidos a medidas restritivas para controle da COVID-19 de acordo com a Lei Nº 13.979 de 6 de fevereiro de 2020, onde os indivíduos são submetidos ao isolamento social, de meios de transporte, mercadorias ou encomendas postais e a quarentena que restringe atividades, separa indivíduos suspeitos de contaminação de outros indivíduos que não estejam doentes.

%\url{http://www.escritacientifica.com/}

%%%%%%%%%%%%%%%%%%%%%%%%%%%%%%%%%%%%%%%%%%%%%%%%%%%%%%%%%%%%%%%%%%%%%%%%%%%%%%%%
%%%%%%%%%%%%%%%%%%%%%%%%%%%%%%%%%%%%%%%%%%%%%%%%%%%%%%%%%%%%%%%%%%%%%%%%%%%%%%%%
%%%%%%%%%%%%%%%%%%%%%%%%%%%%%%%%%%%%%%%%%%%%%%%%%%%%%%%%%%%%%%%%%%%%%%%%%%%%%%%%
\section{Questão de pesquisa}%
As perguntas fundamentais que foram base para essa pesquisa.

É possível construir uma simulação baseada em agentes da pandemia do COVID-19 em um determinado espaço geográfico ?  Essa simulação corresponde de fato ao que aconteceu ? Alguma dificuldade que não pode ser contornada ?


%%%%%%%%%%%%%%%%%%%%%%%%%%%%%%%%%%%%%%%%%%%%%%%%%%%%%%%%%%%%%%%%%%%%%%%%%%%%%%%%
%%%%%%%%%%%%%%%%%%%%%%%%%%%%%%%%%%%%%%%%%%%%%%%%%%%%%%%%%%%%%%%%%%%%%%%%%%%%%%%%
%%%%%%%%%%%%%%%%%%%%%%%%%%%%%%%%%%%%%%%%%%%%%%%%%%%%%%%%%%%%%%%%%%%%%%%%%%%%%%%%
\section{Relevância e Justificativa}%

%Finalidade: nosso trabalho é uma pesquisa básica e estratégica, ou seja, pode ser usado como base para resolver outros problemas%

%Objetivo da nossa pesquisa é descritiva e exploratória pois levantamos dados que não estão em livros%

%Abordagem quali-quantitativa pois usamos métodos estatísticos e simulações%

%Método hipotético-dedutivo pois usamos a simulação para verificar o comportamento da população sob restrições da pandemia de COVID-19%

Simulações baseadas em agentes pode ser entendida como simulação de agentes individuais, os quais geram algo mais complexo a partir da ação de cada indivíduo, dessa forma um fenômeno em grande escala pode ser mais facilmente entendido se observado cada agente de forma individual.

Um bom exemplo seria a movimentação de um cardume de peixes, trânsito em uma cidade, disseminação de uma doença e voo de pássaros. Inclusive esse último possui um modelo muito bem elaborado na ferramenta Netlogo, uma linguagem de programação e um ambiente de desenvolvimento integrado baseada em agentes, que utiliza-se de 3 regras “alinhamento”, “coesão” e “separação” na movimentação individual de cada pássaro para obter o movimento coordenado de um bando de pássaros.

Neste trabalho será utilizado a ferramenta GAMA devido a sua praticidade tanto com dados georeferenciados quanto por ser uma linguagem de alto nível, a qual possibilita a criação intuitiva de agentes. Além disso possui diversas interfaces, tanto 2D quanto 3D, para a visualização de resultados com painel controláveis pelo usuário.

A história mostra que a humanidade já passou por outras pandemias e por conta da atual globalização existe uma enorme possibilidade de acontecer outra como a atual da COVID-19. Por isso é importante a construção de modelos para analisar parâmetros e padrões das pandemias com objetivo de ajudar tanto no controle da atual quanto na redução de danos de futuras pandemias. 

Este trabalho baseado na linguagem GAMA busca obter um modelo ao qual forneça dados aproximados ao que de fato aconteceu em uma região de Brasília, com isso é possível buscar meios e novas perspectivas para o enfrentamento da COVID-19.

**(alterar depois) Também será mostrado nesse trabalho a tentativa de elaboração do mesmo modelo, porém em uma região diferente de Brasília com pouco acesso a informação de dados geográficos, qualitativos e quantitativos.


%%%%%%%%%%%%%%%%%%%%%%%%%%%%%%%%%%%%%%%%%%%%%%%%%%%%%%%%%%%%%%%%%%%%%%%%%%%%%%%%
%%%%%%%%%%%%%%%%%%%%%%%%%%%%%%%%%%%%%%%%%%%%%%%%%%%%%%%%%%%%%%%%%%%%%%%%%%%%%%%%
%%%%%%%%%%%%%%%%%%%%%%%%%%%%%%%%%%%%%%%%%%%%%%%%%%%%%%%%%%%%%%%%%%%%%%%%%%%%%%%%

\section{Objetivos Gerais e Específicos}%
Como objetivo geral nesse trabalho é desenvolver uma modelagem e simulação espacialmente explícita sobre o impacto da COVID-19, em região próxima ao Distrito Federal, utilizando-se uma linguagem baseada em agentes.

Como objetivos específicos: 
\begin{enumerate}
\item Identificar ferramentas para obter dados de um sistema de informação geográfica livre e aberta;
\item Identificar ferramentas para as modelagens e simulações;
\item Identificar regiões as quais são possíveis obter dados e desenvolver um modelo;
\item Construir o modelo baseado na região selecionada;
\item Comparar o resultado obtido no modelo com os dados apresentados na região.
\end{enumerate}

%%%%%%%%%%%%%%%%%%%%%%%%%%%%%%%%%%%%%%%%%%%%%%%%%%%%%%%%%%%%%%%%%%%%%%%%%%%%%%%%
%%%%%%%%%%%%%%%%%%%%%%%%%%%%%%%%%%%%%%%%%%%%%%%%%%%%%%%%%%%%%%%%%%%%%%%%%%%%%%%%
%%%%%%%%%%%%%%%%%%%%%%%%%%%%%%%%%%%%%%%%%%%%%%%%%%%%%%%%%%%%%%%%%%%%%%%%%%%%%%%%
\section{Metodologia}%

Nessa pesquisa foram analisadas duas regiões distintas do DF, para tal foi utilizado as ferramentas QGIS para mapeamento da região e GAMA para a modelagem. 

Para criar um modelo no GAMA que utilize-se de uma área geográfica é necessário possuir ou criar os arquivos shapefile, no caso de Santa Luzia por ser um assentamento irregular, poucos dados estavam disponíveis.

Tendo isso em vista, a região de Santa Luzia foi mapeada manualmente no OpenStreetMap e posteriormente foi feita a exportação do arquivo para o QGIS, com isso foi possível gerar o shapefile para exportar para o GAMA. 

Na construção do modelo baseado em agentes é necessário definir quais serão os agentes, seus parâmetros, comportamentos e interações com outros agentes. Observando-se isso foi realizado um estudo sobre a região para identificar os principais agentes e suas relações na comunidade, locais de lazer, escolas, hospitais, comércio, são agentes que possuem um impacto significativo na simulação.

Para auxiliar na elaboração da modelagem da COVID-19 na Santa Luzia foi utilizado o COMOKIT (Kit de modelagem da COVID-19) feito no GAMA para comparar diversos tipos de intervenções possíveis, durante as adaptações do modelo foram encontradas diversas dificuldade em obter dados de infectados e curados pela Covid na região da Santa Luzia o que tornou inviável a continuação da modelagem.

Com esse obstáculo foi necessário procurar uma nova região a qual pudesse fornecer de forma confiável esses dados que posteriormente ficou definido como Condomínio RK.

No Condomínio RK já foi possível obter dados como shapefile e diversos dados relacionados a COVID na região, dessa forma foi possível fazer ajustes e adaptações no COMOKIT e verificar que o modelo obteve um comportamento similar ao ocorrido de fato na região.

%%%%%%%%%%%%%%%%%%%%%%%%%%%%%%%%%%%%%%%%%%%%%%%%%%%%%%%%%%%%%%%%%%%%%%%%%%%%%%%%
%%%%%%%%%%%%%%%%%%%%%%%%%%%%%%%%%%%%%%%%%%%%%%%%%%%%%%%%%%%%%%%%%%%%%%%%%%%%%%%%
%%%%%%%%%%%%%%%%%%%%%%%%%%%%%%%%%%%%%%%%%%%%%%%%%%%%%%%%%%%%%%%%%%%%%%%%%%%%%%%%




%\section{Normas CIC}
% \href{http://monografias.cic.unb.br/dspace/normasGerais.pdf}{Política de Publicação de Monografias e Dissertações no Repositório Digital do CIC}%
% \href{http://monografias.cic.unb.br/dspace/}{Repositório do Departamento de Ciência da Computação da UnB}

% \href{http://bdm.bce.unb.br/}{Biblioteca Digital de Monografias de Graduação e Especialização}