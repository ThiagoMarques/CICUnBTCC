%%%%%%%%%%%%%%%%%%%%%%%%%%%%%%%%%%%%%%%%
% Classe do documento
%%%%%%%%%%%%%%%%%%%%%%%%%%%%%%%%%%%%%%%%

% Opções:
%  - Graduação: bacharelado|engenharia|licenciatura
%  - Pós-graduação: [qualificacao], mestrado|doutorado, ppca|ppginf

% \documentclass[engenharia]{UnB-CIC}%
\documentclass[licenciatura]{UnB-CIC}%

\usepackage{pdfpages}% incluir PDFs, usado no apêndice
\usepackage{authblk}%
\usepackage{indentfirst}


%%%%%%%%%%%%%%%%%%%%%%%%%%%%%%%%%%%%%%%%
% Informações do Trabalho
%%%%%%%%%%%%%%%%%%%%%%%%%%%%%%%%%%%%%%%%
\orientador{\prof \dr Jorge Henrique Cabral Fernandes}{CIC/UnB}%
%\coorientador{\prof \dr José Ralha}{CIC/UnB}
%\\coordenador[a]{\prof[a] \dr[a] Ada Lovelace}{Bibliothèque universelle de Genève}%
\diamesano{30}{maio}{2022}%

%\\membrobanca{\prof \dr Donald Knuth}{Stanford University}%
%\\membrobanca{\dr Leslie Lamport}{Microsoft Research}%

\author[1]{Ricardo A. Rodrigues}
\author[2]{Thiago S. Marques}
%\author[2]{Corresponding Author\thanks{email@2nduniversity.com}}
\affil[1-2]{}



\titulo{Simulação multiagente da epidemia de COVID-19 usando o COMOKIT/GAMA em dois territórios no Distrito Federal: Os casos do Condomínio RK e da Santa Luzia}%

\palavraschave{COMOKIT, COVID-19, políticas de saúde, modelagem e simulação}%
\keywords{COMOKIT, COVID-19, health policy, modeling and simulation}%


%%%%%%%%%%%%%%%%%%%%%%%%%%%%%%%%%%%%%%%%
% Texto
%%%%%%%%%%%%%%%%%%%%%%%%%%%%%%%%%%%%%%%%
\begin{document}%
    \capitulo{1_Introducao}{Introdução}%
    \capitulo{2_Referencial Teórico}{Referencial Teórico}%
    \capitulo{3_Construção das Simulações}{Construção das Simulações}%
    \capitulo{4_Execução das Simulações e Coleta de Dados}{Execução das Simulações e Coleta de Dados}%
    \capitulo{5_Comparações entre Simulação e Realidade}{Comparações entre Simulação e Realidade}%
    \capitulo{6_Utilidade Potencial da Simulação em Políticas de Saúde Pública}{Utilidade Potencial da Simulação em Políticas de Saúde Pública}%
    \capitulo{7_Discussões}{Discussões}%
    \capitulo{8_Conclusão}{Conclusões}%

    \apendice{Apendice_Fichamento}{Fichamento de Artigo Científico}%
    \anexo{Anexo1}{Documentação Original (parcial)}%
\end{document}%